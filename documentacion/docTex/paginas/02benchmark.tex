\section{Benchmark}

Este paso lo hemos completado para conocer las \textbf{soluciones actuales} en el mundo, conocer las fortalezas, debilidades, las formas de trabajo, conocer de sus interfaces y procesos. Con esta información tendremos un panorama de lo que ya existe y podremos hacer un camino de usuario para lograr una experiencia más amigable de nuestra aplicación. 

\begin{table}[h!]
\centering
\begin{tabular}{|>{\centering\arraybackslash}m{2.5cm}|>{\centering\arraybackslash}m{3cm}|>{\centering\arraybackslash}m{2.5cm}|>{\centering\arraybackslash}m{3cm}|>{\centering\arraybackslash}m{3cm}|} % Definir ancho de columna y centrar texto
\hline
\textbf{} & \textbf{FixMyStreet (Reino Unido)} & \textbf{SeeClickFix (EE.UU.)} & \textbf{Mejora tu Ciudad} & \textbf{Colab.re (Brasil)} \\ \hline
\textbf{Marcación en mapa} & Sí (OpenStreetMap) & Sí (Google Maps) & Sí (Google Maps) & Sí (Leaflet/OpenStreetMap) \\ \hline
\textbf{Adjuntar fotos} & Sí (1+ imágenes) & Sí (hasta 5 imágenes) & Sí (fotos y videos) & Sí (3+ imágenes) \\ \hline
\textbf{Descripción breve} & Texto libre & Campos predefinidos + texto & Texto libre & Texto + etiquetas \\ \hline
\textbf{Actualización de estado} & No (solo creador/autoridades) & Sí (usuarios verificados) & Sí (usuarios y autoridades) & Sí (usuarios registrados) \\ \hline
\textbf{Visualización de incidentes} & Mapa con filtros & Mapa interactivo con capas & Mapa con íconos & Mapa con filtros y heatmaps \\ \hline
\textbf{API del mapa} & OpenLayers + OpenStreetMap & Google Maps API + Esri & Google Maps API & Leaflet API + OpenStreetMap \\ \hline
\textbf{Tecnología} & Perl, PostgreSQL, OpenLayers & React, Node.js, AWS &  MySQL/Firebase  & JavaScript, Python, PostgreSQL \\ \hline
\textbf{Disponibilidad} & Reino Unido, Australia & EE.UU., Canadá &  España, México  & Brasil, Argentina, Uruguay \\ \hline
\end{tabular}
\caption{Comparativa de plataformas de gestión de incidentes urbanos}
\label{tab:comparativa}
\end{table}

\subsection*{Enlaces a las páginas}
\begin{minipage}[t]{0.48\textwidth}
\begin{itemize}
    \item \href{https://www.fixmystreet.com/}{FixMyStreet (Reino Unido)}
    \item \href{https://play.google.com/store/apps/details?id=com.seeclickfix.ma.android&hl=es_MX}{SeeClickFix (EE.UU.)}
\end{itemize}
\end{minipage}
\hfill
\begin{minipage}[t]{0.48\textwidth}
\begin{itemize}
    \item \href{https://mejoratuciudad.org/}{Mejora tu Ciudad (España)}
    \item \href{https://www.colab.com.br/sou-cidadao/}{Colab.re (Brasil)}
\end{itemize}
\end{minipage}


\begin{table}[H]
\centering
\begin{tabular}{|>{\centering\arraybackslash}m{2.7cm}|>{\centering\arraybackslash}m{3cm}|>{\centering\arraybackslash}m{3cm}|>{\centering\arraybackslash}m{3cm}|>{\centering\arraybackslash}m{3cm}|>{\centering\arraybackslash}m{3cm}|}
\hline
\textbf{} & \textbf{FixMyStreet (UK)} & \textbf{SeeClickFix (EEUU)} & \textbf{Mejora tu Ciudad (ES)} & \textbf{Colab.re (BR)} \\ \hline

\textbf{Fortalezas} & 
Integración autoridades \newline Open-source \newline PostgreSQL robusto & 
Interfaz intuitiva \newline Colaboración real-time \newline Escalabilidad AWS & 
Multimedia completo \newline Flexibilidad descripciones & 
Visualización avanzada \newline Stack moderno \newline Foco LATAM \\ \hline

\textbf{Debilidades} & 
Actualizaciones limitadas \newline Alcance reducido & 
APIs pagas \newline Límite imágenes & 
Tecnología oculta \newline Cobertura indefinida & 
Baja visibilidad global \newline Registro obligatorio \\ \hline

\textbf{Base de datos} & 
PostgreSQL & 
AWS RDS \newline(MySQL/PG) & 
MySQL/Firebase & 
PostgreSQL \\ \hline

\textbf{Propuesta de valor} & 
Comunicación simple \newline ciudadano-gobierno & 
Transparencia mediante \newline reportes verificados & 
Empoderamiento ciudadano \newline con evidencia visual & 
Soluciones basadas \newline en datos LATAM \\ \hline

\end{tabular}
\caption{Comparativa de puntos específicos}
\label{tab:comparativa-optimizada}
\end{table}


\subsection*{Conclusiones}

Basado en el análisis de la competencia, se identificaron prácticas clave que se aplicarán en \textit{Bumper}:

\begin{itemize}
    \item En FixMyStreet y Colab.re, los filtros en el mapa mejoran la navegación; implementaremos filtros por estatus en React con Leaflet.
    \item SeeClickFix permite actualizar el estatus con usuarios verificados; habilitaremos esta función para el creador del reporte vía Spring Boot.
    \item Mejora tu Ciudad destaca por subir multimedia; integraremos la carga de 1-3 fotos, almacenadas en Supabase o el servidor.
    \item Colab.re usa Leaflet y OpenStreetMap con éxito; adoptaremos esta API gratuita para el mapa interactivo.
    \item Identificamos en las vistas que componentes como barras laterales y dropmenus mejorar la experiencia de los usuarios al realizar reportes.
\end{itemize}

Estas decisiones aprovechan las fortalezas de la competencia, adaptándolas a nuestra pila tecnológica (Spring Boot, React, PostgreSQL) para una solución eficiente y diferenciada.