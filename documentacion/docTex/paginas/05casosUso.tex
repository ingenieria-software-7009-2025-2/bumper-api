\section{Casos de Uso}

\begin{multicols}{2}

% --- CASO DE USO 1: HACER LOGIN ---
\subsection*{CU1: Autenticación de Usuario}
\textbf{Actores:} Ciudadano, Sistema\\
\textbf{Flujo Principal:}
\begin{enumerate}
    \item Usuario accede a la página de login (RF1)
    \item Ingresa email y contraseña registrados
    \item Sistema verifica credenciales en base de datos (RNF1)
    \item Redirección al dashboard principal (RNF4)
\end{enumerate}

\textbf{Flujo Alternativo:}
\begin{itemize}
    \item Credenciales incorrectas:
    \begin{enumerate}
        \item Sistema muestra error específico
        \item Ofrece recuperación de contraseña vía email
    \end{enumerate}
    \item Cuenta no verificada:
    \begin{enumerate}
        \item Sistema bloquea acceso
        \item Reenvía enlace de verificación
    \end{enumerate}
\end{itemize}

% --- CASO DE USO 2: REPORTAR INCIDENTE ---
\subsection*{CU2: Reportar Incidente Urbano}
\textbf{Actores:} Ciudadano, Sistema\\
\textbf{Flujo Principal:}
\begin{enumerate}
    \item El usuario inicia sesión (RF1) o se registra
    \item Selecciona Reportar Incidente desde el mapa (RF3)
    \item Marca ubicación en el mapa con geolocalización asistida (RNF2)
    \item Completa formulario con: categoría, título, descripción y fotos (RF2)
    \item Sistema valida y almacena el reporte en la base de datos (RNF5)
    \item Muestra página de confirmación con código único (RF5)
\end{enumerate}

% --- CASO DE USO 3: EDITAR CASO URBANO ---
\subsection*{CU3: Modificar Reporte Existente}
\textbf{Actores:} Ciudadano, Sistema\\
\textbf{Flujo Principal:}
\begin{enumerate}
    \item Usuario accede a Mis Reportes en su perfil (RF2)
    \item Selecciona incidente no publicado para editar
    \item Modifica campos permitidos: descripción, fotos, etiquetas (RF2)
    \item Sistema valida cambios y actualiza timestamp (RNF5)
    \item Muestra versión actualizada con historial de modificaciones
\end{enumerate}

\textbf{Flujo Alternativo:}
\begin{itemize}
    \item Intento de editar reporte publicado:
    \begin{enumerate}
        \item Sistema bloquea cambios directos
        \item Ofrece crear nueva versión como actualización
    \end{enumerate}
\end{itemize}

% --- CASO DE USO 4: INTERACCIÓN CON MAPA ---
\subsection*{CU4: Explorar Incidentes en Mapa}
\textbf{Actores:} Usuario (registrado/invitado), Sistema\\
\textbf{Flujo Principal:}
\begin{enumerate}
    \item Usuario visualiza mapa principal (RF3)
    \item Aplica filtros por categoría/fecha (RF4)
    \item Haz clic en marcador para ver detalle
    \item Sistema carga pop-up con:
    \begin{itemize}
        \item Galería de fotos (RNF2)
        \item Estado actualizado
        \item Botones de interacción
    \end{itemize}
\end{enumerate}

\textbf{Flujo Alternativo:}
\begin{itemize}
    \item Filtros sin resultados:
    \begin{itemize}
        \item Sistema sugiere ajustar parámetros
        \item Muestra heatmap de zonas problemáticas
    \end{itemize}
\end{itemize}

% --- CASO DE USO 5: COMENTAR REPORTE ---
\subsection*{CU5: Añadir Comentario Público}
\textbf{Actores:} Usuario Registrado, Sistema\\
\textbf{Flujo Principal:}
\begin{enumerate}
    \item Usuario selecciona incidente en mapa/listado (RF4)
    \item Clic en Añadir Comentario
    \item Escribe mensaje
    \item Sistema analiza contenido (RNF1):
    \begin{itemize}
        \item Filtra lenguaje ofensivo
        \item Detecta spam
    \end{itemize}
    \item Publica comentario
\end{enumerate}

\textbf{Flujo Alternativo:}
\begin{itemize}
    \item Comentario rechazado:
    \begin{enumerate}
        \item Sistema muestra políticas de comunidad
        \item Ofrece editar el contenido
    \end{enumerate}
\end{itemize}

\end{multicols}