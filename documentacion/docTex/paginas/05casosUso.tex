\section{Casos de Uso}

\begin{multicols}{2}

% --- CASO DE USO 1: REGISTRO USUARIO ---
\subsection*{CU1: Registro de Usuario}
\begin{itemize}
    \item \textbf{Actores:} Ciudadano, Sistema
    \item \textbf{Precondiciones:} El usuario no se encuentra registrado
    \item \textbf{Postcondiciones:} El usuario es registrado y puede iniciar sesión con sus credenciales
\end{itemize}
\textbf{Flujo Principal:}
\begin{enumerate}
    \item Usuario accede a la página de registro (RF1)
    \item Ingresa datos del formulario: nombre, apellido, correo y contraseña
    \item Sistema valida la información: completa y el correo no se encuentra registrado 
    \item Redirección al dashboard principal (RNF4)
\end{enumerate}

\textbf{Flujo Alternativo:}
\begin{itemize}
    \item Formulario incompleto:
    \begin{enumerate}
        \item Sistema muestra ``Por favor, llena este campo'' en el espacio sin llenar
    \end{enumerate}
    \item Formato de correo invalido:
    \begin{enumerate}
        \item Sistema muestra ``Correo invalido''
    \end{enumerate}
    \item Correo ya registrado:
    \begin{enumerate}
        \item Sistema muestra ``Error al registrar el usuario''
    \end{enumerate}
\end{itemize}

\textbf{Flujo Excepcional:}
\begin{itemize}
    \item Servidor no disponible:
    \begin{enumerate}
        \item Sistema muestra error ...
    \end{enumerate}
\end{itemize}

% --- CASO DE USO 2: HACER LOGIN ---
\subsection*{CU2: Iniciar Sesión}
\begin{itemize}
    \item \textbf{Actores:} Ciudadano, Sistema
    \item \textbf{Precondiciones:} El usuario se encuentra registrado 
    \item \textbf{Postcondiciones:} El usuario accede a la página
\end{itemize}
\textbf{Flujo Principal:}
\begin{enumerate}
    \item Usuario accede a la página de login (RF1)
    \item Ingresa email y contraseña registrados
    \item Sistema verifica credenciales en base de datos (RNF1)
    \item Redirección al dashboard principal (RNF4)
\end{enumerate}

\textbf{Flujo Alternativo:}
\begin{itemize}
    \item Credenciales incorrectas:
    \begin{enumerate}
        \item Sistema muestra error específico
        \item Ofrece recuperación de contraseña vía email
    \end{enumerate}
    \item Cuenta no verificada:
    \begin{enumerate}
        \item Sistema bloquea acceso
        \item Reenvía enlace de verificación
    \end{enumerate}
\end{itemize}

\textbf{Flujo Excepcional:}
\begin{itemize}
    \item Servidor no disponible:
    \begin{enumerate}
        \item Sistema muestra error ...
    \end{enumerate}
\end{itemize}

% --- CASO DE USO 3: ACTUALIZAR DATOS USUARIO ---
\subsection*{CU3: Actualizar datos de Usuario}
\begin{itemize}
    \item \textbf{Actores:} Ciudadano, Sistema
    \item \textbf{Precondiciones:} El usuario tiene su sesión iniciada
    \item \textbf{Postcondiciones:} Los datos del usuario han sido modificados
\end{itemize}
\textbf{Flujo Principal:}
\begin{enumerate}
    \item Usuario accede al apartado ``Perfil''
    \item Sistema muestra un apartado con campos que contienen la información del usuario
    \item El usuario modifica los datos de alguno de los campos
    \item El sistema valida la información y actualiza
    \item El sistema muestra mensaje: ``Cambios guardados correctamente''
\end{enumerate}

\textbf{Flujo Alternativo:}
\begin{itemize}
    \item La nueva información es invalida:
    \begin{enumerate}
        \item Sistema muestra mensaje de error
    \end{enumerate}
\end{itemize}

\textbf{Flujo Excepcional:}
\begin{itemize}
    \item Servidor no disponible:
    \begin{enumerate}
        \item Sistema muestra error...
    \end{enumerate}
\end{itemize}

% --- CASO DE USO 4: CONSULTAR INFORMACIÖN USUARIO ---
\subsection*{CU4: Consultar información del      Usuario}
\begin{itemize}
    \item \textbf{Actores:} Ciudadano, Sistema
    \item \textbf{Precondiciones:} El usuario tiene su sesión iniciada
    \item \textbf{Postcondiciones:} Se muestran los datos del propio usuario y sus reportes realizados.
\end{itemize}
\textbf{Flujo Principal:}
\begin{enumerate}
    \item Usuario accede al apartado ``Perfil''
    \item Sistema muestra un apartado con campos que contienen la información del usuario y un aparatado llamado ``Incidentes Reportados'' con los incidente reportados por el usuario
\end{enumerate}

\textbf{Flujo Alternativo:}
\begin{itemize}
    \item El usuario no ha reportado incidentes:
    \begin{enumerate}
        \item Sistema muestra dentro del apartado ``Incidentes Reportados'' el mensaje ``No hay incidentes reportados aún'' 
    \end{enumerate}
\end{itemize}

\textbf{Flujo Excepcional:}
\begin{itemize}
    \item Servidor no disponible:
    \begin{enumerate}
        \item Sistema muestra error...
    \end{enumerate}
\end{itemize}

% --- CASO DE USO 5: CERRAR SESIÓN ---
\subsection*{CU5: Cerrar Sesión}
\begin{itemize}
    \item \textbf{Actores:} Ciudadano, Sistema
    \item \textbf{Precondiciones:} El usuario tiene su sesión iniciada
    \item \textbf{Postcondiciones:} El usuario cierra su sesión, vuelve a la pantalla de login
\end{itemize}
\textbf{Flujo Principal:}
\begin{enumerate}
    \item Usuario presiona en el apartado ``Cerrar Sesión''
    \item El sistema cierra sesión, reenvia a la pantalla de login
\end{enumerate}

\textbf{Flujo Excepcional:}
\begin{itemize}
    \item Servidor no disponible:
    \begin{enumerate}
        \item Sistema muestra error...
    \end{enumerate}
\end{itemize}

% --- CASO DE USO 6: REPORTAR INCIDENTE ---
\subsection*{CU6: Reportar Incidente Urbano}
\begin{itemize}
    \item \textbf{Actores:} Ciudadano, Sistema
    \item \textbf{Precondiciones:} El usuario tiene su sesión iniciada
    \item \textbf{Postcondiciones:} Se registra un nuevo incidente
\end{itemize}
\begin{enumerate}
    \item El usuario se encuentra en la pantalla principal
    \item Selecciona ``Reportar Incidente'' desde el mapa (RF3)
    \item Establece ubicación mediante alguno de los métodos: geolocalización asistida (RNF2), formulario, selección en el mapa
    \item Completa formulario de descripción con: categoría, título, descripción y fotos (RF2)
    \item Sistema valida y almacena el reporte en la base de datos (RNF5)
    \item Muestra página de confirmación con código único (RF5)
\end{enumerate}

% --- CASO DE USO 7: MODIFICAR INCIDENTE URBANO ---
\subsection*{CU7: Modificar Reporte Existente}
\begin{itemize}
    \item \textbf{Actores:} Ciudadano, Sistema
    \item \textbf{Precondiciones:} El usuario tiene su sesión iniciada
    \item \textbf{Postcondiciones:} Información de incidente actualizada 
\end{itemize}
\textbf{Flujo Principal:}
\begin{enumerate}
    \item Usuario accede a Mis Reportes en su perfil (RF2)
    \item Selecciona incidente no publicado para editar
    \item Modifica campos permitidos: descripción, fotos, etiquetas (RF2)
    \item Sistema valida cambios y actualiza timestamp (RNF5)
    \item Muestra versión actualizada con historial de modificaciones
\end{enumerate}

\textbf{Flujo Alternativo:}
\begin{itemize}
    \item Intento de editar reporte publicado:
    \begin{enumerate}
        \item Sistema bloquea cambios directos
        \item Ofrece crear nueva versión como actualización
    \end{enumerate}
\end{itemize}

% --- CASO DE USO 8: INTERACCIÓN CON MAPA ---
\subsection*{CU8: Explorar Incidentes en Mapa}
\textbf{Actores:} Usuario (registrado/invitado), Sistema\\
\textbf{Flujo Principal:}
\begin{enumerate}
    \item Usuario visualiza mapa principal (RF3)
    \item Aplica filtros por categoría/fecha (RF4)
    \item Haz clic en marcador para ver detalle
    \item Sistema carga pop-up con:
    \begin{itemize}
        \item Galería de fotos (RNF2)
        \item Estado actualizado
        \item Botones de interacción
    \end{itemize}
\end{enumerate}

\textbf{Flujo Alternativo:}
\begin{itemize}
    \item Filtros sin resultados:
    \begin{itemize}
        \item Sistema sugiere ajustar parámetros
        \item Muestra heatmap de zonas problemáticas
    \end{itemize}
\end{itemize}

% --- CASO DE USO 9: COMENTAR REPORTE ---
\subsection*{CU9: Añadir Comentario Público}
\textbf{Actores:} Usuario Registrado, Sistema\\
\textbf{Flujo Principal:}
\begin{enumerate}
    \item Usuario selecciona incidente en mapa/listado (RF4)
    \item Clic en Añadir Comentario
    \item Escribe mensaje
    \item Sistema analiza contenido (RNF1):
    \begin{itemize}
        \item Filtra lenguaje ofensivo
        \item Detecta spam
    \end{itemize}
    \item Publica comentario
\end{enumerate}

\textbf{Flujo Alternativo:}
\begin{itemize}
    \item Comentario rechazado:
    \begin{enumerate}
        \item Sistema muestra políticas de comunidad
        \item Ofrece editar el contenido
    \end{enumerate}
\end{itemize}

% --- CASO DE USO 10: ACTUALIZAR ESTADO DE INCIDENTE ---
\subsection*{CU10: Actualizar Estado de Incidente}
\begin{itemize}
    \item \textbf{Actores:} Ciudadano, Sistema
    \item \textbf{Precondiciones:} El usuario tiene su sesión iniciada
    \item \textbf{Postcondiciones:} Cambia el estado del incidente seleccionado a Resuelto
\end{itemize}
\textbf{Flujo Principal:}
\begin{enumerate}
    \item Usuario selecciona incidente en mapa/listado (RF4)
    \item Clic en Marcar como Resuelto
    \item Sistema solicita en formulario: Subir 1+fotos de evidencia y descripción
    \item Se visualiza sobre el incidente, la solicitud de cambio de estado, con la opción de aprobar
    \item Sistema en espera de concluir verificación de la solicitud, estado se encuentra En Proceso
    \item Sistema valida solicitud, estado cambia a Resuelto
    \item Se añade nuevo boton de interacción para discrepar: Esto no está resuelto
\end{enumerate}

\textbf{Flujo Alternativo:}
\begin{itemize}
    \item Solicitud rechazada:
    \begin{enumerate}
        \item Sistema manda notificación ``Actualización de estado rechazado, Incidente ? (enlace a incidente) ''
    \end{enumerate}
\end{itemize}

% --- CASO DE USO 11: APROBAR SOLICITUD DE INCIDENTE RESUELTO ---
\subsection*{CU11: Aprobar Cambio de Estado}
\begin{itemize}
    \item \textbf{Actores:} Ciudadano, Sistema
    \item \textbf{Precondiciones:} El usuario tiene su sesión iniciada y seleccionó un incidente con solicitud de cambio de estado
    \item \textbf{Postcondiciones:} Aumenta el número de aprobaciones de la solicitud de cambio de estado de un incidente
\end{itemize}
\textbf{Flujo Principal:}
\begin{enumerate}
    \item Usuario selecciona incidente en mapa/listado (RF4)
    \item Clic en aprobar, sobre una solicitud de cambio de estado
    \item Sistema aumenta el número de aprobaciones de la solicitud
\end{enumerate}

% --- CASO DE USO 12: DISCREPAR INCIDENTE RESUELTO ---
\subsection*{CU12: Discrepar sobre Incidente Resuelto}
\begin{itemize}
    \item \textbf{Actores:} Ciudadano, Sistema
    \item \textbf{Precondiciones:} El usuario tiene su sesión iniciada y seleccionó un incidente con estado Resuelto
    \item \textbf{Postcondiciones:} Cambia el estado del incidente seleccionado a No Resuelto
\end{itemize}
\textbf{Flujo Principal:}
\begin{enumerate}
    \item Usuario selecciona incidente en mapa/listado (RF4)
    \item Clic en Esto no está resuelto
    \item Sistema solicita en formulario de disputa: Seleccionar motivo, adjuntar foto de prueba (opcional)
    \item Se cambia el estado a En Proceso
    \item Sistema valida información del formulario, el estado cambia a No Resuelto
     
\end{enumerate}

\textbf{Flujo Alternativo:}
\begin{itemize}
    \item Solicitud de disputa rechazada:
    \begin{enumerate}
        \item Sistema manda notificación ``Disputa de Incidente Resuelto rechazada, Incidente ? (enlace a incidente) ''
    \end{enumerate}
\end{itemize}

% --- CASO DE USO 13: EXPLORAR MENSAJE/NOTIFICACIONES ---
\subsection*{CU13: Explorar Mensajes}
\begin{itemize}
    \item \textbf{Actores:} Ciudadano, Sistema
    \item \textbf{Precondiciones:} El usuario tiene su sesión iniciada
    \item \textbf{Postcondiciones:} Se visualizan las notificaciones recibidas por el usuario
\end{itemize}
\textbf{Flujo Principal:}
\begin{enumerate}
    \item Usuario selecciona el apartado Mensajes
    \item Sistema muestra pantalla de Mensajes
    \item Se visualiza un apartado con los encabezados de los mensajes entrantes/notificaciones
    \item Usuario hace clic en algun mensaje
    \item Sistema muestra el cuerpo completo del mensaje
\end{enumerate}

\textbf{Flujo Alternativo:}
\begin{itemize}
    \item No se han recibido mensajes:
    \begin{enumerate}
        \item Sistema muestra texto Sin mensajes disponibles aún
    \end{enumerate}
\end{itemize}

% --- CASO DE USO 14: Eliminar incidente ---
\subsection*{CU14: Eliminar incidente}
\begin{itemize}
    \item \textbf{Actores:} Administrador, Sistema
    \item \textbf{Precondiciones:} El usuario tiene su sesión iniciada como administrador
    \item \textbf{Postcondiciones:} Se borra del sistema el incidente seleccionado
\end{itemize}
\textbf{Flujo Principal:}
\begin{enumerate}
     \item Administrador selecciona incidente en mapa/listado (RF4)
    \item Clic en Eliminar
    \item Sistema muestra mensaje de confirmación
    \item Administrador confirma eliminación
    \item Sistema borra completamente el incidente
    \item Sistema muestra el mensaje Eliminación exitosa
\end{enumerate}

\end{multicols}