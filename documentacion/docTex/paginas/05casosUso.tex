\section{Casos de Uso}

\subsection*{CU1: Hacer Login}
\textbf{Actores:} Usuario, Sistema\\
\textbf{Flujo Principal:}
\begin{enumerate}
    \item El usuario accede a la página de inicio de sesión. (RF1)
    \item Ingresa su correo electrónico y contraseña.
    \item El sistema verifica las credenciales.
    \item Si las credenciales son correctas, el usuario es redirigido al mapa principal. (RF5)
\end{enumerate}
\textbf{Flujo Alternativo:}
\begin{itemize}
    \item \textit{Credenciales incorrectas:}
    \begin{enumerate}
        \item El sistema muestra un mensaje de error indicando que las credenciales no son válidas.
        \item Ofrece un enlace para recuperar la contraseña por correo.
    \end{enumerate}
\end{itemize}

\subsection*{CU2: Ingreso del Usuario}
\textbf{Actores:} Ciudadano, Sistema\\
\textbf{Flujo Principal:}
\begin{enumerate}
    \item El usuario accede a la página de inicio de sesión. (RF1)
    \item Ingresa su correo electrónico y contraseña.
    \item El sistema verifica las credenciales.
    \item El usuario es redirigido al mapa principal. (RF5)
\end{enumerate}
\textbf{Flujo Alternativo:}
\begin{itemize}
    \item \textit{Credenciales incorrectas:}
    \begin{enumerate}
        \item El sistema muestra un mensaje de error.
        \item Ofrece un enlace para recuperar la contraseña por correo.
    \end{enumerate}
\end{itemize}

\subsection*{CU3: Reportar Incidente Urbano}
\textbf{Actores:} Ciudadano, Sistema\\
\textbf{Flujo Principal:}
\begin{enumerate}
    \item El usuario inicia sesión. (RF1)
    \item Hace clic en ``Reportar Incidente'' desde el mapa principal. (RF3)
    \item Marca la ubicación en el mapa o escribe una dirección. (RF3)
    \item Completa un formulario con tipo de incidente, descripción y hasta 3 fotos. (RF2)
    \item Confirma el reporte tras ver una vista previa.
    \item El sistema muestra el reporte en el mapa con un código único. (RF5)
\end{enumerate}
\textbf{Flujo Alternativo:}
\begin{itemize}
    \item \textit{Sin inicio de sesión:}
    \begin{enumerate}
        \item El sistema redirige al usuario a la página de inicio de sesión antes de continuar.
    \end{enumerate}
\end{itemize}

\subsection*{CU4: Modificar Estatus de Reporte}
\textbf{Actores:} Ciudadano (creador del reporte), Sistema\\
\textbf{Flujo Principal:}
\begin{enumerate}
    \item El usuario inicia sesión y selecciona uno de sus reportes en el mapa. (RF2)
    \item Abre la ventana del reporte y elige un nuevo estatus (``En proceso'' o ``Resuelto''). (RF2)
    \item Confirma el cambio.
    \item El sistema actualiza el estatus y muestra la versión modificada en el mapa. (RF5)
\end{enumerate}
\textbf{Flujo Alternativo:}
\begin{itemize}
    \item \textit{Usuario no creador:}
    \begin{enumerate}
        \item El sistema no muestra opciones de edición y limita la interacción a solo visualización.
    \end{enumerate}
\end{itemize}

\subsection*{CU5: Explorar Incidentes en Mapa}
\textbf{Actores:} Usuario (registrado o invitado), Sistema\\
\textbf{Flujo Principal:}
\begin{enumerate}
    \item El usuario accede al mapa principal y ve los marcadores de incidentes. (RF3)
    \item Usa filtros en la barra lateral para mostrar incidentes por estatus o tipo. (RF4)
    \item Hace clic en un marcador para ver detalles.
    \item El sistema muestra una ventana con fotos, estatus y tipo de incidente. (RF5)
\end{enumerate}
\textbf{Flujo Alternativo:}
\begin{itemize}
    \item \textit{Filtros sin resultados:}
    \begin{enumerate}
        \item El sistema indica que no hay incidentes y sugiere ajustar los filtros.
    \end{enumerate}
\end{itemize}

\subsection*{CU6: Cerrar Sesión}
\textbf{Actores:} Usuario, Sistema\\
\textbf{Flujo Principal:}
\begin{enumerate}
    \item El usuario hace clic en el botón de "Cerrar Sesión" desde cualquier página del sistema. (RF6)
    \item El sistema cierra la sesión del usuario y lo redirige a la página de inicio de sesión. (RF1)
\end{enumerate}
\textbf{Flujo Alternativo:}
\begin{itemize}
    \item \textit{Error en el cierre de sesión:}
    \begin{enumerate}
        \item El sistema muestra un mensaje de error indicando que no se pudo cerrar la sesión.
        \item El usuario puede intentar nuevamente o contactar soporte técnico.
    \end{enumerate}
\end{itemize}

\subsection*{CU7: Ver Detalles de un Incidente}
\textbf{Actores:} Usuario (registrado o invitado), Sistema\\
\textbf{Flujo Principal:}
\begin{enumerate}
    \item El usuario accede al mapa principal. (RF3)
    \item Hace clic en un marcador de incidente en el mapa.
    \item El sistema muestra una ventana emergente con los detalles del incidente, incluyendo:
    \begin{itemize}
        \item Tipo de incidente.
        \item Ubicación.
        \item Estado actual (Pendiente, En proceso, Resuelto).
        \item Fotos asociadas al incidente.
        \item Fecha y hora del reporte.
    \end{itemize}
\end{enumerate}
\textbf{Flujo Alternativo:}
\begin{itemize}
    \item \textit{Incidente sin fotos:}
    \begin{enumerate}
        \item El sistema muestra un mensaje indicando que no hay fotos disponibles para este incidente.
    \end{enumerate}
\end{itemize}

\subsection*{CU8: Editar Perfil de Usuario}
\textbf{Actores:} Usuario, Sistema\\
\textbf{Flujo Principal:}
\begin{enumerate}
    \item El usuario inicia sesión en el sistema. (RF1)
    \item Accede a la sección de "Perfil" desde el menú principal. (RF10)
    \item Visualiza su información actual, como nombre, apellido, correo electrónico y contraseña.
    \item Realiza cambios en los campos permitidos (por ejemplo, nombre o contraseña).
    \item Confirma los cambios haciendo clic en "Guardar". (RF11)
    \item El sistema valida los datos ingresados y actualiza la información en la base de datos.
    \item El sistema muestra un mensaje de confirmación indicando que los cambios se han guardado correctamente.
\end{enumerate}
\textbf{Flujo Alternativo:}
\begin{itemize}
    \item \textit{Datos inválidos:}
    \begin{enumerate}
        \item El sistema muestra un mensaje de error indicando qué campos necesitan corrección.
        \item El usuario corrige los datos y vuelve a intentar guardar los cambios.
    \end{enumerate}
\end{itemize}