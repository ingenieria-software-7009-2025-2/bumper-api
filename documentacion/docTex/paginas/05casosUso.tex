\section{Casos de Uso de Bumper}

\subsection{Acceso y Registro de Usuario}
\textbf{Actor principal:} Usuario no autenticado\\
\textbf{Propósito:} Permitir que un nuevo usuario se registre en el sistema para poder reportar incidentes.\\
\textbf{Precondición:} El usuario no debe estar registrado previamente.\\
\textbf{Flujo principal:}
\begin{itemize}
    \item El usuario accede a la página de registro.
    \item Ingresa nombre, apellido, correo electrónico y contraseña.
    \item El sistema valida el formato del correo y la unicidad.
    \item Si los datos son válidos, el sistema crea la cuenta y muestra un mensaje de éxito.
    \item Si hay errores, el sistema informa el motivo (correo inválido o ya registrado).
\end{itemize}
\textbf{Postcondición:} El usuario queda registrado y puede iniciar sesión.

\subsection{Inicio de Sesión}
\textbf{Actor principal:} Usuario registrado\\
\textbf{Propósito:} Permitir que un usuario autenticado acceda a las funcionalidades avanzadas del sistema.\\
\textbf{Precondición:} El usuario debe estar registrado.\\
\textbf{Flujo principal:}
\begin{itemize}
    \item El usuario accede a la página de inicio de sesión.
    \item Ingresa su correo electrónico y contraseña.
    \item El sistema valida las credenciales.
    \item Si son correctas, el usuario accede a su cuenta.
    \item Si son incorrectas, el sistema muestra un mensaje de error.
\end{itemize}
\textbf{Postcondición:} El usuario queda autenticado en el sistema.

\subsection{Creación de Reporte de Incidente}
\textbf{Actor principal:} Usuario autenticado\\
\textbf{Propósito:} Permitir que un usuario reporte un nuevo incidente en la ciudad.\\
\textbf{Precondición:} El usuario debe estar autenticado.\\
\textbf{Flujo principal:}
\begin{itemize}
    \item El usuario selecciona la opción ``Reportar Incidente''.
    \item Completa el formulario con ubicación, tipo de incidente, descripción y fotos.
    \item El sistema valida los datos y registra el incidente.
    \item El sistema muestra un mensaje de confirmación y el incidente aparece en el mapa.
\end{itemize}
\textbf{Postcondición:} El incidente queda registrado y visible en el mapa.

\subsection{Consulta y Gestión de Mis Reportes}
\textbf{Actor principal:} Usuario autenticado\\
\textbf{Propósito:} Permitir que el usuario consulte, actualice el estatus o elimine los incidentes que ha reportado.\\
\textbf{Precondición:} El usuario debe estar autenticado y tener al menos un incidente reportado.\\
\textbf{Flujo principal:}
\begin{itemize}
    \item El usuario accede a la sección ``Mis reportes''.
    \item El sistema muestra la lista de incidentes reportados por el usuario.
    \item El usuario puede seleccionar un incidente para ver detalles.
    \item El usuario puede cambiar el estatus del incidente (por ejemplo, a ``Resuelto'').
    \item El usuario puede eliminar un incidente propio si lo desea.
\end{itemize}
\textbf{Postcondición:} El usuario gestiona sus reportes; los cambios se reflejan en el sistema.

\subsection{Visualización de Incidentes en el Mapa}
\textbf{Actor principal:} Usuario autenticado o invitado\\
\textbf{Propósito:} Permitir que cualquier usuario visualice los incidentes reportados en el mapa.\\
\textbf{Precondición:} El sistema debe tener incidentes registrados.\\
\textbf{Flujo principal:}
\begin{itemize}
    \item El usuario accede a la página principal.
    \item El sistema muestra un mapa con íconos representando los incidentes (por tipo: basura, bache, etc.).
    \item El usuario puede hacer clic en un ícono para ver detalles básicos (fotos, estatus, tipo).
    \item El usuario puede filtrar los incidentes por tipo o estatus usando la barra lateral.
\end{itemize}
\textbf{Postcondición:} El usuario visualiza la información pública de los incidentes.

\subsection{Cierre de Sesión}
\textbf{Actor principal:} Usuario autenticado\\
\textbf{Propósito:} Permitir que el usuario cierre su sesión de manera segura.\\
\textbf{Precondición:} El usuario debe estar autenticado.\\
\textbf{Flujo principal:}
\begin{itemize}
    \item El usuario selecciona la opción ``Cerrar sesión''.
    \item El sistema invalida la sesión y muestra la pantalla de inicio.
\end{itemize}
\textbf{Postcondición:} El usuario vuelve al estado de invitado.

\subsection{Cambio de Contraseña}
\textbf{Actor principal:} Usuario autenticado\\
\textbf{Propósito:} Permitir que un usuario autenticado cambie su contraseña desde su perfil.\\
\textbf{Precondición:} El usuario debe estar autenticado.\\
\textbf{Flujo principal:}
\begin{itemize}
    \item El usuario accede a su perfil
    \item Selecciona la opción \textit{Cambiar contraseña}.
    \item Ingresa su contraseña actual, la nueva contraseña y la confirmación de la nueva contraseña.
    \item El sistema valida que la contraseña actual sea correcta y que las nuevas contraseñas coincidan.
    \item El sistema actualiza la contraseña del usuario.
\end{itemize}
\textbf{Postcondición:} El usuario puede acceder nuevamente con la nueva contraseña.\\


\subsection{Descarga de Reporte de Incidente como Imagen para Redes Sociales}
\textbf{Actor principal:} Usuario autenticado o invitado\\
\textbf{Propósito:} Permitir que el usuario descargue una imagen con la información clave de un incidente para compartir en redes sociales.\\
\textbf{Precondición:} El sistema debe tener incidentes registrados.\\
\textbf{Flujo principal:}
\begin{itemize}
    \item El usuario accede a la página de detalles de un incidente.
    \item Selecciona la opción ``Descargar imagen para compartir''.
    \item El sistema genera una imagen con:
    \begin{itemize}
        \item Una foto del incidente (si está disponible).
        \item El tipo de incidente y una breve descripción.
        \item El estatus actual.
        \item Un texto de concientización o llamado a la acción.
        \item El logo de la aplicación y un enlace al sitio web.
    \end{itemize}
    \item El usuario descarga la imagen en formato JPG o PNG.
\end{itemize}
\textbf{Postcondición:} El usuario tiene una imagen lista para compartir en redes sociales, promoviendo la conciencia ciudadana y el uso de la aplicación.
\textit{Nota: Este caso de uso requiere la implementación de una función para generar la imagen con los datos del incidente.}