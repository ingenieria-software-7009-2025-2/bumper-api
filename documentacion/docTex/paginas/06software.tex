\section{Tecnología Usada}

\subsection*{Backend}
\begin{itemize}
    \item \textbf{Lenguaje de Programación: Kotlin}
    \begin{itemize}
        \item \textit{Razón de uso:} Kotlin es un lenguaje moderno, conciso y seguro que mejora la productividad del desarrollo al reducir el código repetitivo y minimizar errores comunes (como null pointer exceptions). Su interoperabilidad con Java permite aprovechar bibliotecas existentes, mientras que su sintaxis clara facilita el mantenimiento del código.
        \item \textit{Ventaja para Bumper:} Ideal para implementar una API RESTful robusta (RF2, RF3, RF5, RF6) que gestione autenticación, cierre de sesión y operaciones sobre reportes, así como la generación de imágenes para compartir incidentes (RF7). Su eficiencia y escalabilidad facilitan la integración de nuevas funciones como actualización de contraseña (RF8) y notificaciones futuras.
    \end{itemize}
    \item \textbf{Framework: Spring Boot}
    \begin{itemize}
        \item \textit{Razón de uso:} Spring Boot simplifica la creación de aplicaciones backend con configuraciones automáticas, integración nativa con bases de datos y soporte para seguridad. Su arquitectura basada en microservicios permite manejar solicitudes simultáneas con baja latencia (RNF2).
        \item \textit{Ventaja para Bumper:} Facilita la gestión de reportes, autenticación de usuarios y la lógica de negocio para la descarga de imágenes de incidentes (RF7). Su modularidad permite crecer hacia funcionalidades como recuperación de contraseña o notificaciones (RF8, sugerido).
    \end{itemize}
\end{itemize}

\subsection*{Frontend}
\begin{itemize}
    \item \textbf{React}
    \begin{itemize}
        \item \textit{Razón de uso:} React es una biblioteca de JavaScript que permite construir interfaces dinámicas y responsivas (RNF4) mediante componentes reutilizables. Su enfoque en el estado y la renderización eficiente soporta interacciones en tiempo real, como filtros en el mapa y vistas previas de reportes.
        \item \textit{Ventaja para Bumper:} Perfecto para integrar un mapa interactivo con íconos personalizados por tipo de incidente (RF4, RF5), una barra lateral dinámica y la opción de descargar imágenes de reportes para redes sociales (RF7). Su ecosistema (e.g., react-leaflet) simplifica la integración con mapas y la experiencia móvil (RNF3).
    \end{itemize}
\end{itemize}

\subsection*{Base de Datos}
\begin{itemize}
    \item \textbf{PostgreSQL en Supabase}
    \begin{itemize}
        \item \textit{Razón de uso:} PostgreSQL es un sistema de base de datos relacional robusto y de código abierto, ideal para almacenar datos estructurados como reportes con ubicaciones, estatus y fotos. Supabase añade una capa de facilidad con almacenamiento de archivos (fotos) y autenticación integrada, reduciendo la complejidad inicial.
        \item \textit{Ventaja para Bumper:} Garantiza la persistencia de los reportes y usuarios, permite consultas rápidas para filtros y visualización en el mapa (RF4, RF5), y soporta la escalabilidad para más usuarios y datos (RNF6). El almacenamiento de imágenes en Supabase facilita la generación de imágenes para compartir (RF7).
    \end{itemize}
\end{itemize}

\subsection*{API para Mapas}
\begin{itemize}
    \item \textbf{Leaflet con OpenStreetMap (Opción Principal)}
    \begin{itemize}
        \item \textit{Razón de uso:} Leaflet es una biblioteca ligera y de código abierto para mapas interactivos, combinada con OpenStreetMap, una fuente de datos gratuita y global. Esto permite visualizar y marcar incidentes (RF4, RF5) sin costos asociados, a diferencia de APIs comerciales.
        \item \textit{Ventaja para Bumper:} Su integración con React (via react-leaflet) asegura un mapa fluido y personalizable, con íconos diferenciados por tipo de incidente, filtros dinámicos y baja latencia (RNF2). Es ideal para un MVP escalable y económico (RNF6), con soporte para marcadores, popups y visualización de detalles para descarga de imágenes (RF7).
    \end{itemize}
\end{itemize}