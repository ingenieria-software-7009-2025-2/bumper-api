\section{Tecnología Usada}


\subsection*{Backend}
\begin{itemize}
    \item \textbf{Lenguaje de Programación: Kotlin}
    \begin{itemize}
        \item \textit{Razón de uso:} Kotlin es un lenguaje moderno, conciso y seguro que mejora la productividad del desarrollo al reducir el código repetitivo y minimizar errores comunes (como null pointer exceptions). Su interoperabilidad con Java permite aprovechar bibliotecas existentes, mientras que su sintaxis clara facilita el mantenimiento del código.
        \item \textit{Ventaja para Bumper:} Ideal para implementar una API RESTful robusta (RF2, RF3) que gestione autenticación (RF1) y modificaciones de reportes (CU3), asegurando un backend eficiente y fácil de escalar (RNF5).
    \end{itemize}
    \item \textbf{Framework: Spring Boot}
    \begin{itemize}
        \item \textit{Razón de uso:} Spring Boot simplifica la creación de aplicaciones backend con configuraciones automáticas, integración nativa con bases de datos y soporte para seguridad. Su arquitectura basada en microservicios permite manejar solicitudes simultáneas con baja latencia (RNF2).
        \item \textit{Ventaja para Bumper:} Facilita la gestión de reportes (RF2) y la autenticación de usuarios (CU1), ofreciendo un entorno estable y modular que puede crecer con nuevas funcionalidades, como notificaciones o integración con autoridades (RNF5).
    \end{itemize}
\end{itemize}

\subsection*{Frontend}
\begin{itemize}
    \item \textbf{React}
    \begin{itemize}
        \item \textit{Razón de uso:} React es una biblioteca de JavaScript que permite construir interfaces dinámicas y responsivas (RNF4) mediante componentes reutilizables. Su enfoque en el estado y la renderización eficiente soporta interacciones en tiempo real, como filtros en el mapa (RF4) y vistas previas de reportes (CU2).
        \item \textit{Ventaja para Bumper:} Perfecto para integrar un mapa interactivo (RF3) y una barra lateral dinámica (CU4), asegurando una experiencia de usuario intuitiva y adaptable a dispositivos móviles (RNF4). Además, su ecosistema (e.g., react-leaflet) simplifica la integración con mapas.
    \end{itemize}
\end{itemize}

\subsection*{Base de Datos}
\begin{itemize}
    \item \textbf{PostgreSQL en Supabase}
    \begin{itemize}
        \item \textit{Razón de uso:} PostgreSQL es un sistema de base de datos relacional robusto y de código abierto, ideal para almacenar datos estructurados como reportes con ubicaciones, estatus y fotos (RF2). Supabase añade una capa de facilidad con almacenamiento de archivos (fotos) y autenticación integrada, reduciendo la complejidad inicial.
        \item \textit{Ventaja para Bumper:} Garantiza la persistencia de los reportes (CU2, CU3) y permite consultas rápidas para filtros (RF4), con escalabilidad para soportar más usuarios y datos (RNF5). El uso de Supabase ofrece un nivel gratuito inicial, optimizando costos para el prototipo y el que este activo siempre nos da cierta ventaja.
    \end{itemize}
\end{itemize}

\subsection*{API para Mapas}
\begin{itemize}
    \item \textbf{Leaflet con OpenStreetMap (Opción Principal)}
    \begin{itemize}
        \item \textit{Razón de uso:} Leaflet es una biblioteca ligera y de código abierto para mapas interactivos, combinada con OpenStreetMap, una fuente de datos gratuita y global. Esto permite visualizar y marcar incidentes (RF3) sin costos asociados, a diferencia de APIs comerciales.
        \item \textit{Ventaja para Bumper:} Su integración con React (via react-leaflet) asegura un mapa fluido y personalizable (CU4), con filtros dinámicos (RF4) y baja latencia (RNF2). Es ideal para un MVP escalable y económico (RNF5), con soporte para marcadores y popups (CU2, CU3).
    \end{itemize}
    \item \textbf{Google Maps API (Alternativa)}
    \begin{itemize}
        \item \textit{Razón de uso:} Ofrece funcionalidades avanzadas como autocompletado de direcciones y alta calidad visual, útil si se prioriza una experiencia premium en el futuro.
        \item \textit{Ventaja para Bumper:} Podría mejorar la precisión de ubicaciones (CU2), pero su costo (crédito limitado de \$200/mes) lo hace menos viable para la fase inicial frente a Leaflet.
    \end{itemize}
\end{itemize}