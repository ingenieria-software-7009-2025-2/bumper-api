\section{Introducción}

Proyecto: \textbf{Bumper}

Aplicación web para el registro y gestión de incidentes urbanos en la Ciudad de México.

\subsection{Propósito}

Esta aplicación web está pensada para permitir a los usuarios \textbf{registrar}, \textbf{visualizar} y \textbf{gestionar} \textbf{incidentes urbanos}, tales como baches, luminarias descompuestas, obstáculos en la vía pública, basura en banquetas, entre otros. El objetivo principal es brindar una \textbf{herramienta colaborativa} para mejorar la comunicación y participación entre los ciudadanos sobre problemas en la ciudad. 

\subsection{Alcance}

Esta plataforma web permite a ciudadanos, comerciantes y organizaciones vecinales registrar y gestionar incidentes en las calles de sus colonias, utilizando una aplicación móvil sencilla de usar para mejorar la participación comunitaria, optimizar recursos y fomentar una ciudad más segura y limpia.

\subsection{El cliente pidió}

\begin{itemize}
    \item \textbf{Registro de incidentes}
    \begin{itemize}
        \item Permitir a los usuarios marcar la ubicación del incidente en un mapa interactivo.
        \item Adjuntar una o más fotografías que respalden el incidente reportado.
        \item Registrar una breve descripción del problema.
    \end{itemize}
    \item \textbf{Actualización de estado del incidente}
    \begin{itemize}
        \item Permitir que cualquier usuario (no solo el creador del reporte) actualice el estado del incidente a "Resuelto", siempre que adjunte pruebas fotográficas que validen la solución.
    \end{itemize}
    \item \textbf{Visualización de incidentes}
    \begin{itemize}
        \item Mostrar todos los incidentes en un mapa interactivo, categorizados por tipo y estado (Reportado, En proceso, Resuelto).
    \end{itemize}
\end{itemize}
