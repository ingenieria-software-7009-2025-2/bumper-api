\section*{Documentación de la API REST}


% ================ USUARIOS ================

% --- 1. Registrar Usuario ---
\subsection*{\faServer\ 1. Registrar Usuario}
\begin{tcolorbox}[endpoint]
    \textbf{POST} \texttt{/v1/users/create}\\
    Registra un nuevo usuario en el sistema.
\end{tcolorbox}

\begin{tcolorbox}[request]
    \textbf{URL:} \textcolor{urlColor}{\texttt{http://localhost:8080/v1/users/create}}\\
    \textbf{Método:} \textcolor{methodColor}{\texttt{POST}}\\
    \textbf{Headers:} \textcolor{headerColor}{\texttt{Content-Type: application/json}}\\
    \textbf{Body:}
    \begin{minted}[frame=single, bgcolor=gray!10, fontsize=\small]{json}
{
    "nombre": "Juan",
    "apellido": "Pérez",
    "correo": "juan@example.com",
    "password": "secreto123"
}
    \end{minted}
\end{tcolorbox}

\begin{tcolorbox}[response]
    \textbf{Status:} 201 Created\\
    \textbf{Body:}
    \begin{minted}[frame=single, bgcolor=gray!10, fontsize=\small]{json}
{
    "mensaje": "Usuario registrado exitosamente",
    "usuario": {
        "id": 1,
        "nombre": "Juan",
        "apellido": "Pérez",
        "correo": "juan@example.com",
        "token": "inactivo",
        "numeroIncidentes": 0,
        "fechaRegistro": "2025-05-04T12:00:00"
    }
}
    \end{minted}
\end{tcolorbox}

\begin{tcolorbox}[response]
    \textbf{Status:} 400 Bad Request\\
    \textbf{Body (correo inválido):}
    \begin{minted}[frame=single, bgcolor=gray!10, fontsize=\small]{json}
{
    "mensaje": "Correo electrónico inválido"
}
    \end{minted}
\end{tcolorbox}

\begin{tcolorbox}[response]
    \textbf{Status:} 400 Bad Request\\
    \textbf{Body (validación):}
    \begin{minted}[frame=single, bgcolor=gray!10, fontsize=\small]{json}
{
    "mensaje": "Error de validación"
}
    \end{minted}
\end{tcolorbox}

\begin{tcolorbox}[response]
    \textbf{Status:} 500 Internal Server Error\\
    \textbf{Body:}
    \begin{minted}[frame=single, bgcolor=gray!10, fontsize=\small]{json}
{
    "mensaje": "Error interno al registrar usuario"
}
    \end{minted}
\end{tcolorbox}


% --- 2. Obtener Usuario por Correo ---
\subsection*{\faServer\ 2. Obtener Usuario por Correo}
\begin{tcolorbox}[endpoint]
    \textbf{GET} \texttt{/v1/users/correo/\{correo\}}\\
    Obtiene los datos de un usuario por su correo electrónico.
\end{tcolorbox}

\begin{tcolorbox}[request]
    \textbf{URL:} \textcolor{urlColor}{\texttt{http://localhost:8080/v1/users/correo/juan@example.com}}\\
    \textbf{Método:} \textcolor{methodColor}{\texttt{GET}}
\end{tcolorbox}

\begin{tcolorbox}[response]
    \textbf{Status:} 200 OK\\
    \textbf{Body:}
    \begin{minted}[frame=single, bgcolor=gray!10, fontsize=\small]{json}
{
    "mensaje": "Usuario encontrado",
    "usuario": {
        "id": 1,
        "nombre": "Juan",
        "apellido": "Pérez",
        "correo": "juan@example.com",
        "token": "activo",
        "numeroIncidentes": 0,
        "fechaRegistro": "2025-05-04T12:00:00"
    }
}
    \end{minted}
\end{tcolorbox}

\begin{tcolorbox}[response]
    \textbf{Status:} 404 Not Found\\
    \textbf{Body:}
    \begin{minted}[frame=single, bgcolor=gray!10, fontsize=\small]{json}
{
    "mensaje": "Usuario no encontrado"
}
    \end{minted}
\end{tcolorbox}

\begin{tcolorbox}[response]
    \textbf{Status:} 500 Internal Server Error\\
    \textbf{Body:}
    \begin{minted}[frame=single, bgcolor=gray!10, fontsize=\small]{json}
{
    "mensaje": "Error al buscar usuario"
}
    \end{minted}
\end{tcolorbox}

% --- 3. Obtener Usuario por ID ---
\subsection*{\faServer\ 3. Obtener Usuario por ID}
\begin{tcolorbox}[endpoint]
    \textbf{GET} \texttt{/v1/users/\{id\}}\\
    Obtiene los datos de un usuario por su ID.
\end{tcolorbox}

\begin{tcolorbox}[request]
    \textbf{URL:} \textcolor{urlColor}{\texttt{http://localhost:8080/v1/users/1}}\\
    \textbf{Método:} \textcolor{methodColor}{\texttt{GET}}
\end{tcolorbox}

\begin{tcolorbox}[response]
    \textbf{Status:} 200 OK\\
    \textbf{Body:}
    \begin{minted}[frame=single, bgcolor=gray!10, fontsize=\small]{json}
{
    "mensaje": "Usuario encontrado",
    "usuario": {
        "id": 1,
        "nombre": "Juan",
        "apellido": "Pérez",
        "correo": "juan@example.com",
        "token": "activo",
        "numeroIncidentes": 0,
        "fechaRegistro": "2025-05-04T12:00:00"
    }
}
    \end{minted}
\end{tcolorbox}

\begin{tcolorbox}[response]
    \textbf{Status:} 404 Not Found\\
    \textbf{Body:}
    \begin{minted}[frame=single, bgcolor=gray!10, fontsize=\small]{json}
{
    "mensaje": "Usuario no encontrado"
}
    \end{minted}
\end{tcolorbox}

\begin{tcolorbox}[response]
    \textbf{Status:} 500 Internal Server Error\\
    \textbf{Body:}
    \begin{minted}[frame=single, bgcolor=gray!10, fontsize=\small]{json}
{
    "mensaje": "Error al buscar usuario"
}
    \end{minted}
\end{tcolorbox}

% --- 4. Login Usuario ---
\subsection*{\faServer\ 4. Login Usuario}
\begin{tcolorbox}[endpoint]
    \textbf{POST} \texttt{/v1/users/login}\\
    Inicia sesión de usuario.
\end{tcolorbox}

\begin{tcolorbox}[request]
    \textbf{URL:} \textcolor{urlColor}{\texttt{http://localhost:8080/v1/users/login}}\\
    \textbf{Método:} \textcolor{methodColor}{\texttt{POST}}\\
    \textbf{Headers:} \textcolor{headerColor}{\texttt{Content-Type: application/json}}\\
    \textbf{Body:}
    \begin{minted}[frame=single, bgcolor=gray!10, fontsize=\small]{json}
{
    "correo": "juan@example.com",
    "password": "secreto123"
}
    \end{minted}
\end{tcolorbox}

\begin{tcolorbox}[response]
    \textbf{Status:} 200 OK\\
    \textbf{Body:}
    \begin{minted}[frame=single, bgcolor=gray!10, fontsize=\small]{json}
{
    "mensaje": "Login exitoso",
    "usuario": {
        "id": 1,
        "nombre": "Juan",
        "apellido": "Pérez",
        "correo": "juan@example.com",
        "token": "activo",
        "numeroIncidentes": 0,
        "fechaRegistro": "2025-05-04T12:00:00"
    }
}
    \end{minted}
\end{tcolorbox}

\begin{tcolorbox}[response]
    \textbf{Status:} 401 Unauthorized\\
    \textbf{Body:}
    \begin{minted}[frame=single, bgcolor=gray!10, fontsize=\small]{json}
{
    "mensaje": "Credenciales inválidas"
}
    \end{minted}
\end{tcolorbox}

\begin{tcolorbox}[response]
    \textbf{Status:} 500 Internal Server Error\\
    \textbf{Body:}
    \begin{minted}[frame=single, bgcolor=gray!10, fontsize=\small]{json}
{
    "mensaje": "Error en el servidor"
}
    \end{minted}
\end{tcolorbox}

% --- 5. Logout Usuario ---
\subsection*{\faServer\ 5. Logout Usuario}
\begin{tcolorbox}[endpoint]
    \textbf{POST} \texttt{/v1/users/logout}\\
    Cierra la sesión del usuario.
\end{tcolorbox}

\begin{tcolorbox}[request]
    \textbf{URL:} \textcolor{urlColor}{\texttt{http://localhost:8080/v1/users/logout}}\\
    \textbf{Método:} \textcolor{methodColor}{\texttt{POST}}\\
    \textbf{Headers:} \textcolor{headerColor}{\texttt{Content-Type: application/json}}\\
    \textbf{Body:}
    \begin{minted}[frame=single, bgcolor=gray!10, fontsize=\small]{json}
{
    "correo": "juan@example.com"
}
    \end{minted}
\end{tcolorbox}

\begin{tcolorbox}[response]
    \textbf{Status:} 200 OK\\
    \textbf{Body:}
    \begin{minted}[frame=single, bgcolor=gray!10, fontsize=\small]{json}
{
    "mensaje": "Sesión cerrada correctamente"
}
    \end{minted}
\end{tcolorbox}

\begin{tcolorbox}[response]
    \textbf{Status:} 404 Not Found\\
    \textbf{Body:}
    \begin{minted}[frame=single, bgcolor=gray!10, fontsize=\small]{json}
{
    "mensaje": "Usuario no encontrado"
}
    \end{minted}
\end{tcolorbox}

\begin{tcolorbox}[response]
    \textbf{Status:} 500 Internal Server Error\\
    \textbf{Body:}
    \begin{minted}[frame=single, bgcolor=gray!10, fontsize=\small]{json}
{
    "mensaje": "Error al cerrar sesión"
}
    \end{minted}
\end{tcolorbox}

% --- 6. Actualizar Usuario ---
\subsection*{\faServer\ 6. Actualizar Usuario}
\begin{tcolorbox}[endpoint]
    \textbf{PUT} \texttt{/v1/users}\\
    Actualiza los datos de un usuario.
\end{tcolorbox}

\begin{tcolorbox}[request]
    \textbf{URL:} \textcolor{urlColor}{\texttt{http://localhost:8080/v1/users}}\\
    \textbf{Método:} \textcolor{methodColor}{\texttt{PUT}}\\
    \textbf{Headers:} \textcolor{headerColor}{\texttt{Content-Type: application/json}}\\
    \textbf{Body:}
    \begin{minted}[frame=single, bgcolor=gray!10, fontsize=\small]{json}
{
    "id": 1,
    "nombre": "Juan",
    "apellido": "Pérez",
    "correo": "juan@example.com",
    "password": "nuevoPassword123",
    "token": "activo",
    "numeroIncidentes": 0
}
    \end{minted}
\end{tcolorbox}

\begin{tcolorbox}[response]
    \textbf{Status:} 200 OK\\
    \textbf{Body:}
    \begin{minted}[frame=single, bgcolor=gray!10, fontsize=\small]{json}
{
    "mensaje": "Usuario actualizado correctamente",
    "usuario": {
        "id": 1,
        "nombre": "Juan",
        "apellido": "Pérez",
        "correo": "juan@example.com",
        "token": "activo",
        "numeroIncidentes": 0,
        "fechaRegistro": "2025-05-04T12:00:00"
    }
}
    \end{minted}
\end{tcolorbox}

\begin{tcolorbox}[response]
    \textbf{Status:} 400 Bad Request\\
    \textbf{Body (correo inválido):}
    \begin{minted}[frame=single, bgcolor=gray!10, fontsize=\small]{json}
{
    "mensaje": "Correo electrónico inválido"
}
    \end{minted}
\end{tcolorbox}

\begin{tcolorbox}[response]
    \textbf{Status:} 400 Bad Request\\
    \textbf{Body (validación):}
    \begin{minted}[frame=single, bgcolor=gray!10, fontsize=\small]{json}
{
    "mensaje": "Error de validación"
}
    \end{minted}
\end{tcolorbox}

\begin{tcolorbox}[response]
    \textbf{Status:} 500 Internal Server Error\\
    \textbf{Body:}
    \begin{minted}[frame=single, bgcolor=gray!10, fontsize=\small]{json}
{
    "mensaje": "Error al actualizar usuario"
}
    \end{minted}
\end{tcolorbox}

% --- 7. Actualizar Contraseña ---
\subsection*{\faServer\ 7. Actualizar Contraseña}
\begin{tcolorbox}[endpoint]
    \textbf{PUT} \texttt{/v1/users/update-password}\\
    Actualiza la contraseña de un usuario.
\end{tcolorbox}

\begin{tcolorbox}[request]
    \textbf{URL:} \textcolor{urlColor}{\texttt{http://localhost:8080/v1/users/update-password}}\\
    \textbf{Método:} \textcolor{methodColor}{\texttt{PUT}}\\
    \textbf{Headers:} \textcolor{headerColor}{\texttt{Content-Type: application/json}}\\
    \textbf{Body:}
    \begin{minted}[frame=single, bgcolor=gray!10, fontsize=\small]{json}
{
    "id": 1,
    "nuevaPassword": "nuevaPassword123"
}
    \end{minted}
\end{tcolorbox}

\begin{tcolorbox}[response]
    \textbf{Status:} 200 OK\\
    \textbf{Body:}
    \begin{minted}[frame=single, bgcolor=gray!10, fontsize=\small]{json}
{
    "mensaje": "Contraseña actualizada correctamente"
}
    \end{minted}
\end{tcolorbox}

\begin{tcolorbox}[response]
    \textbf{Status:} 400 Bad Request\\
    \textbf{Body (no se pudo actualizar):}
    \begin{minted}[frame=single, bgcolor=gray!10, fontsize=\small]{json}
{
    "mensaje": "No se pudo actualizar la contraseña"
}
    \end{minted}
\end{tcolorbox}

\begin{tcolorbox}[response]
    \textbf{Status:} 400 Bad Request\\
    \textbf{Body (validación):}
    \begin{minted}[frame=single, bgcolor=gray!10, fontsize=\small]{json}
{
    "mensaje": "Error de validación"
}
    \end{minted}
\end{tcolorbox}

\begin{tcolorbox}[response]
    \textbf{Status:} 500 Internal Server Error\\
    \textbf{Body:}
    \begin{minted}[frame=single, bgcolor=gray!10, fontsize=\small]{json}
{
    "mensaje": "Error interno al actualizar contraseña"
}
    \end{minted}
\end{tcolorbox}

% ================ INCIDENTES ================

% --- 1. Crear Incidente ---
\subsection*{\faServer\ 1. Crear Incidente}
\begin{tcolorbox}[endpoint]
    \textbf{POST} \texttt{/v1/incidentes/create}\\
    Registra un nuevo incidente en el sistema.
\end{tcolorbox}

\begin{tcolorbox}[request]
    \textbf{URL:} \textcolor{urlColor}{\texttt{http://localhost:8080/v1/incidentes/create}}\\
    \textbf{Método:} \textcolor{methodColor}{\texttt{POST}}\\
    \textbf{Headers:} \textcolor{headerColor}{\texttt{Content-Type: application/json}}\\
    \textbf{Body:}
    \begin{minted}[frame=single, bgcolor=gray!10, fontsize=\small]{json}
{
    "usuarioId": 1,
    "tipoIncidente": "BACHES",
    "ubicacion": "Calle 123, Colonia Centro",
    "latitud": 19.4326,
    "longitud": -99.1332,
    "tipoVialidad": "CALLE"
}
    \end{minted}
\end{tcolorbox}

\begin{tcolorbox}[response]
    \textbf{Status:} 201 Created\\
    \textbf{Body:}
    \begin{minted}[frame=single, bgcolor=gray!10, fontsize=\small]{json}
{
    "mensaje": "Incidente creado exitosamente",
    "incidente": {
        "id": "20250504120000_BAC_CAL",
        "usuarioId": 1,
        "tipoIncidente": "BACHES",
        "ubicacion": "Calle 123, Colonia Centro",
        "latitud": 19.4326,
        "longitud": -99.1332,
        "horaIncidente": "2025-05-04T12:00:00",
        "tipoVialidad": "CALLE",
        "estado": "PENDIENTE",
        "fotos": []
    }
}
    \end{minted}
\end{tcolorbox}

\begin{tcolorbox}[response]
    \textbf{Status:} 404 Not Found\\
    \textbf{Body:}
    \begin{minted}[frame=single, bgcolor=gray!10, fontsize=\small]{json}
{
    "mensaje": "Usuario no encontrado",
    "usuarioId": 1
}
    \end{minted}
\end{tcolorbox}

\begin{tcolorbox}[response]
    \textbf{Status:} 500 Internal Server Error\\
    \textbf{Body:}
    \begin{minted}[frame=single, bgcolor=gray!10, fontsize=\small]{json}
{
    "mensaje": "Error al registrar el incidente",
    "error": "Descripción del error"
}
    \end{minted}
\end{tcolorbox}

% --- 2. Obtener todos los incidentes ---
\subsection*{\faServer\ 2. Obtener todos los incidentes}
\begin{tcolorbox}[endpoint]
    \textbf{GET} \texttt{/v1/incidentes/all}\\
    Obtiene todos los incidentes registrados.
\end{tcolorbox}

\begin{tcolorbox}[request]
    \textbf{URL:} \textcolor{urlColor}{\texttt{http://localhost:8080/v1/incidentes/all}}\\
    \textbf{Método:} \textcolor{methodColor}{\texttt{GET}}
\end{tcolorbox}

\begin{tcolorbox}[response]
    \textbf{Status:} 200 OK\\
    \textbf{Body (con incidentes):}
    \begin{minted}[frame=single, bgcolor=gray!10, fontsize=\small]{json}
{
    "mensaje": "Incidentes encontrados",
    "total": 2,
    "incidentes": [
        {
            "id": "20250504120000_BAC_CAL",
            "usuarioId": 1,
            "tipoIncidente": "BACHES",
            "ubicacion": "Calle 123, Colonia Centro",
            "latitud": 19.4326,
            "longitud": -99.1332,
            "horaIncidente": "2025-05-04T12:00:00",
            "tipoVialidad": "CALLE",
            "estado": "PENDIENTE",
            "fotos": []
        }
        // ...otros incidentes
    ]
}
    \end{minted}
\end{tcolorbox}

\begin{tcolorbox}[response]
    \textbf{Status:} 200 OK\\
    \textbf{Body (sin incidentes):}
    \begin{minted}[frame=single, bgcolor=gray!10, fontsize=\small]{json}
{
    "mensaje": "No se encontraron incidentes registrados",
    "incidentes": []
}
    \end{minted}
\end{tcolorbox}

\begin{tcolorbox}[response]
    \textbf{Status:} 500 Internal Server Error\\
    \textbf{Body:}
    \begin{minted}[frame=single, bgcolor=gray!10, fontsize=\small]{json}
{
    "mensaje": "Error al obtener los incidentes"
}
    \end{minted}
\end{tcolorbox}

% --- 3. Obtener incidentes por usuario ---
\subsection*{\faServer\ 3. Obtener incidentes por usuario}
\begin{tcolorbox}[endpoint]
    \textbf{GET} \texttt{/v1/incidentes/usuario/\{usuarioId\}}\\
    Obtiene todos los incidentes de un usuario específico.
\end{tcolorbox}

\begin{tcolorbox}[request]
    \textbf{URL:} \textcolor{urlColor}{\texttt{http://localhost:8080/v1/incidentes/usuario/1}}\\
    \textbf{Método:} \textcolor{methodColor}{\texttt{GET}}
\end{tcolorbox}

\begin{tcolorbox}[response]
    \textbf{Status:} 200 OK\\
    \textbf{Body:}
    \begin{minted}[frame=single, bgcolor=gray!10, fontsize=\small]{json}
{
    "mensaje": "Incidentes encontrados para el usuario",
    "usuario": {
        "id": 1,
        "nombre": "Juan Pérez"
    },
    "total": 1,
    "incidentes": [
        {
            "id": "20250504120000_BAC_CAL",
            "usuarioId": 1,
            "tipoIncidente": "BACHES",
            "ubicacion": "Calle 123, Colonia Centro",
            "latitud": 19.4326,
            "longitud": -99.1332,
            "horaIncidente": "2025-05-04T12:00:00",
            "tipoVialidad": "CALLE",
            "estado": "PENDIENTE",
            "fotos": [],
            "usuario": {
                "id": 1,
                "nombre": "Juan",
                "apellido": "Pérez"
            }
        }
    ]
}
    \end{minted}
\end{tcolorbox}

\begin{tcolorbox}[response]
    \textbf{Status:} 404 Not Found\\
    \textbf{Body:}
    \begin{minted}[frame=single, bgcolor=gray!10, fontsize=\small]{json}
{
    "mensaje": "Usuario no encontrado"
}
    \end{minted}
\end{tcolorbox}

\begin{tcolorbox}[response]
    \textbf{Status:} 500 Internal Server Error\\
    \textbf{Body:}
    \begin{minted}[frame=single, bgcolor=gray!10, fontsize=\small]{json}
{
    "mensaje": "Error al obtener los incidentes del usuario"
}
    \end{minted}
\end{tcolorbox}

% --- 4. Obtener incidente por ID ---
\subsection*{\faServer\ 4. Obtener incidente por ID}
\begin{tcolorbox}[endpoint]
    \textbf{GET} \texttt{/v1/incidentes/\{id\}}\\
    Obtiene un incidente por su ID.
\end{tcolorbox}

\begin{tcolorbox}[request]
    \textbf{URL:} \textcolor{urlColor}{\texttt{http://localhost:8080/v1/incidentes/20250504120000\_BAC\_CAL}}\\
    \textbf{Método:} \textcolor{methodColor}{\texttt{GET}}
\end{tcolorbox}

\begin{tcolorbox}[response]
    \textbf{Status:} 200 OK\\
    \textbf{Body:}
    \begin{minted}[frame=single, bgcolor=gray!10, fontsize=\small]{json}
{
    "mensaje": "Incidente encontrado",
    "incidente": {
        "id": "20250504120000_BAC_CAL",
        "usuarioId": 1,
        "tipoIncidente": "BACHES",
        "ubicacion": "Calle 123, Colonia Centro",
        "latitud": 19.4326,
        "longitud": -99.1332,
        "horaIncidente": "2025-05-04T12:00:00",
        "tipoVialidad": "CALLE",
        "estado": "PENDIENTE",
        "fotos": []
    }
}
    \end{minted}
\end{tcolorbox}

\begin{tcolorbox}[response]
    \textbf{Status:} 404 Not Found\\
    \textbf{Body:}
    \begin{minted}[frame=single, bgcolor=gray!10, fontsize=\small]{json}
{
    "mensaje": "Incidente no encontrado"
}
    \end{minted}
\end{tcolorbox}

\begin{tcolorbox}[response]
    \textbf{Status:} 500 Internal Server Error\\
    \textbf{Body:}
    \begin{minted}[frame=single, bgcolor=gray!10, fontsize=\small]{json}
{
    "mensaje": "Error al buscar el incidente"
}
    \end{minted}
\end{tcolorbox}

% --- 5. Actualizar estado de incidente ---
\subsection*{\faServer\ 5. Actualizar estado de incidente}
\begin{tcolorbox}[endpoint]
    \textbf{PUT} \texttt{/v1/incidentes/update-status/\{id\}}\\
    Actualiza el estado de un incidente.
\end{tcolorbox}

\begin{tcolorbox}[request]
    \textbf{URL:} \textcolor{urlColor}{\texttt{http://localhost:8080/v1/incidentes/update-status/20250504120000\_BAC\_CAL?usuarioId=1}}\\
    \textbf{Método:} \textcolor{methodColor}{\texttt{PUT}}\\
    \textbf{Headers:} \textcolor{headerColor}{\texttt{Content-Type: application/json}}\\
    \textbf{Body:}
    \begin{minted}[frame=single, bgcolor=gray!10, fontsize=\small]{json}
{
    "estado": "RESUELTO"
}
    \end{minted}
\end{tcolorbox}

\begin{tcolorbox}[response]
    \textbf{Status:} 200 OK\\
    \textbf{Body:}
    \begin{minted}[frame=single, bgcolor=gray!10, fontsize=\small]{json}
{
    "mensaje": "Estado actualizado correctamente",
    "incidente": {
        "id": "20250504120000_BAC_CAL",
        "usuarioId": 1,
        "tipoIncidente": "BACHES",
        "ubicacion": "Calle 123, Colonia Centro",
        "latitud": 19.4326,
        "longitud": -99.1332,
        "horaIncidente": "2025-05-04T12:00:00",
        "tipoVialidad": "CALLE",
        "estado": "RESUELTO",
        "fotos": []
    }
}
    \end{minted}
\end{tcolorbox}

\begin{tcolorbox}[response]
    \textbf{Status:} 403 Forbidden\\
    \textbf{Body:}
    \begin{minted}[frame=single, bgcolor=gray!10, fontsize=\small]{json}
{
    "mensaje": "No tienes permiso para modificar este incidente"
}
    \end{minted}
\end{tcolorbox}

\begin{tcolorbox}[response]
    \textbf{Status:} 404 Not Found\\
    \textbf{Body:}
    \begin{minted}[frame=single, bgcolor=gray!10, fontsize=\small]{json}
{
    "mensaje": "Incidente no encontrado"
}
    \end{minted}
\end{tcolorbox}

\begin{tcolorbox}[response]
    \textbf{Status:} 500 Internal Server Error\\
    \textbf{Body:}
    \begin{minted}[frame=single, bgcolor=gray!10, fontsize=\small]{json}
{
    "mensaje": "Error al actualizar el estado del incidente"
}
    \end{minted}
\end{tcolorbox}

% --- 6. Buscar incidentes cercanos ---
\subsection*{\faServer\ 6. Buscar incidentes cercanos}
\begin{tcolorbox}[endpoint]
    \textbf{GET} \texttt{/v1/incidentes/cercanos}\\
    Busca incidentes cercanos a una ubicación geográfica.
\end{tcolorbox}

\begin{tcolorbox}[request]
    \textbf{URL:} \textcolor{urlColor}{\texttt{http://localhost:8080/v1/incidentes/cercanos?latitud=19.4326\&longitud=-99.1332\&radioKm=5.0}}\\
    \textbf{Método:} \textcolor{methodColor}{\texttt{GET}}
\end{tcolorbox}

\begin{tcolorbox}[response]
    \textbf{Status:} 200 OK\\
    \textbf{Body:}
    \begin{minted}[frame=single, bgcolor=gray!10, fontsize=\small]{json}
{
    "mensaje": "Búsqueda completada",
    "parametros": {
        "latitud": 19.4326,
        "longitud": -99.1332,
        "radioKm": 5.0
    },
    "total": 1,
    "incidentes": [
        {
            "id": "20250504120000_BAC_CAL",
            "usuarioId": 1,
            "tipoIncidente": "BACHES",
            "ubicacion": "Calle 123, Colonia Centro",
            "latitud": 19.4326,
            "longitud": -99.1332,
            "horaIncidente": "2025-05-04T12:00:00",
            "tipoVialidad": "CALLE",
            "estado": "RESUELTO",
            "fotos": []
        }
    ]
}
    \end{minted}
\end{tcolorbox}

\begin{tcolorbox}[response]
    \textbf{Status:} 500 Internal Server Error\\
    \textbf{Body:}
    \begin{minted}[frame=single, bgcolor=gray!10, fontsize=\small]{json}
{
    "mensaje": "Error al buscar incidentes cercanos"
}
    \end{minted}
\end{tcolorbox}

% --- 7. Eliminar incidente ---
\subsection*{\faServer\ 7. Eliminar incidente}
\begin{tcolorbox}[endpoint]
    \textbf{DELETE} \texttt{/v1/incidentes/\{id\}}\\
    Elimina un incidente creado por un usuario.
\end{tcolorbox}

\begin{tcolorbox}[request]
    \textbf{URL:} \textcolor{urlColor}{\texttt{http://localhost:8080/v1/incidentes/20250504120000\_BAC\_CAL?usuarioId=1}}\\
    \textbf{Método:} \textcolor{methodColor}{\texttt{DELETE}}
\end{tcolorbox}

\begin{tcolorbox}[response]
    \textbf{Status:} 200 OK\\
    \textbf{Body:}
    \begin{minted}[frame=single, bgcolor=gray!10, fontsize=\small]{json}
{
    "mensaje": "Incidente eliminado correctamente"
}
    \end{minted}
\end{tcolorbox}

\begin{tcolorbox}[response]
    \textbf{Status:} 403 Forbidden\\
    \textbf{Body:}
    \begin{minted}[frame=single, bgcolor=gray!10, fontsize=\small]{json}
{
    "mensaje": "No tienes permiso para eliminar este incidente"
}
    \end{minted}
\end{tcolorbox}

\begin{tcolorbox}[response]
    \textbf{Status:} 404 Not Found\\
    \textbf{Body:}
    \begin{minted}[frame=single, bgcolor=gray!10, fontsize=\small]{json}
{
    "mensaje": "Incidente no encontrado"
}
    \end{minted}
\end{tcolorbox}

\begin{tcolorbox}[response]
    \textbf{Status:} 500 Internal Server Error\\
    \textbf{Body:}
    \begin{minted}[frame=single, bgcolor=gray!10, fontsize=\small]{json}
{
    "mensaje": "Error al eliminar el incidente",
    "error": "Descripción del error"
}
    \end{minted}
\end{tcolorbox}