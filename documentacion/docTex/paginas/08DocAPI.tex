\section{Documentación API}

\textbf{1. Crear un nuevo usuario}

\textit{Endpoint:} \texttt{POST /v1/users/create}

\textit{Descripción:} Crea un nuevo usuario en la base de datos.

\textit{Solicitud:}
\begin{itemize}
    \item \textbf{URL:} \texttt{http://localhost:8080/v1/users/create}
    \item \textbf{Método:} \texttt{POST}
    \item \textbf{Headers:} \texttt{Content-Type: application/json}
    \item \textbf{Body:}
\end{itemize}

\begin{lstlisting}
{
    "nombre": "Juan",
    "apellido": "Perez",
    "correo": "juan@example.com",
    "password": "1234",
    "token": "inactivo",
    "numeroIncidentes": 0
}
\end{lstlisting}

\textit{Respuesta esperada (éxito, HTTP 201):}
\begin{lstlisting}
{
    "id": 1,
    "nombre": "Juan",
    "apellido": "Perez",
    "correo": "juan@example.com",
    "password": "1234",
    "token": "inactivo",
    "numeroIncidentes": 0,
    "incidentes": []
}
\end{lstlisting}

\textit{Notas:}
\begin{itemize}
    \item El \texttt{id} será generado automáticamente por la base de datos
    \item \texttt{token} y \texttt{numeroIncidentes} tienen valores por defecto
\end{itemize}

\textbf{2. Iniciar sesión (Login)}

\textit{Endpoint:} \texttt{POST /v1/users/login}

\textit{Descripción:} Inicia sesión y cambia el token a \texttt{"activo"}.

\textit{Solicitud:}
\begin{itemize}
    \item \textbf{URL:} \texttt{http://localhost:8080/v1/users/login}
    \item \textbf{Método:} \texttt{POST}
    \item \textbf{Headers:} \texttt{Content-Type: application/json}
    \item \textbf{Body:}
\end{itemize}

\begin{lstlisting}
{
    "correo": "juan@example.com",
    "password": "1234"
}
\end{lstlisting}

\textit{Respuesta esperada (éxito, HTTP 200):}
\begin{lstlisting}
{
    "id": 1,
    "nombre": "Juan",
    "apellido": "Perez",
    "correo": "juan@example.com",
    "password": "1234",
    "token": "activo",
    "numeroIncidentes": 0,
    "incidentes": []
}
\end{lstlisting}

\textit{Notas:}
\begin{itemize}
    \item Credenciales incorrectas devuelven HTTP 401
\end{itemize}

\textbf{3. Crear un nuevo incidente}

\textit{Endpoint:} \texttt{POST /v1/incidentes}

\textit{Descripción:} Crea incidente asociado al usuario.

\textit{Solicitud:}
\begin{itemize}
    \item \textbf{URL:} \texttt{http://localhost:8080/v1/incidentes}
    \item \textbf{Método:} \texttt{POST}
    \item \textbf{Headers:} \texttt{Content-Type: application/json}
    \item \textbf{Body:}
\end{itemize}

\begin{lstlisting}
{
    "usuarioId": 1,
    "tipoIncidente": "BACHES",
    "ubicacion": "Av. Principal 123",
    "tipoVialidad": "AVENIDA"
}
\end{lstlisting}

\textit{Respuesta esperada (éxito, HTTP 201):}
\begin{lstlisting}
{
    "id": 1,
    "usuario": {
        "id": 1,
        "nombre": "Juan",
        "apellido": "Perez",
        "correo": "juan@example.com",
        "password": "1234",
        "token": "activo",
        "numeroIncidentes": 1,
        "incidentes": []
    },
    "tipoIncidente": "BACHES",
    "ubicacion": "Av. Principal 123",
    "horaIncidente": "2025-03-23T10:00:00",
    "tipoVialidad": "AVENIDA"
}
\end{lstlisting}

\textbf{4. Recuperar todos los incidentes}

\textit{Endpoint:} \texttt{GET /v1/incidentes}

\textit{Descripción:} Lista todos los incidentes en la base de datos.

\textit{Solicitud:}
\begin{itemize}
    \item \textbf{URL:} \texttt{http://localhost:8080/v1/incidentes}
    \item \textbf{Método:} \texttt{GET}
    \item \textbf{Headers:} (ninguno requerido)
    \item \textbf{Body:} (ninguno)
\end{itemize}

\textit{Respuesta esperada (éxito, HTTP 200):}
\begin{lstlisting}
[
    {
        "id": 1,
        "usuario": {
            "id": 1,
            "nombre": "Juan",
            "apellido": "Perez",
            "correo": "juan@example.com",
            "password": "1234",
            "token": "activo",
            "numeroIncidentes": 1,
            "incidentes": []
        },
        "tipoIncidente": "BACHES",
        "ubicacion": "Av. Principal 123",
        "horaIncidente": "2025-03-23T10:00:00",
        "tipoVialidad": "AVENIDA"
    }
]
\end{lstlisting}

\textit{Notas:}
\begin{itemize}
    \item Muestra todos los incidentes creados en el sistema
\end{itemize}

\textbf{5. Recuperar incidentes por usuario}

\textit{Endpoint:} \texttt{GET /v1/incidentes/usuario/\{usuarioId\}}

\textit{Descripción:} Lista incidentes asociados a un usuario específico.

\textit{Solicitud:}
\begin{itemize}
    \item \textbf{URL:} \texttt{http://localhost:8080/v1/incidentes/usuario/1}
    \item \textbf{Método:} \texttt{GET}
    \item \textbf{Headers:} (ninguno requerido)
    \item \textbf{Body:} (ninguno)
\end{itemize}

\textit{Respuesta esperada (éxito, HTTP 200):}
\begin{lstlisting}
[
    {
        "id": 1,
        "usuario": {
            "id": 1,
            "nombre": "Juan",
            "apellido": "Perez",
            "correo": "juan@example.com",
            "password": "1234",
            "token": "activo",
            "numeroIncidentes": 1,
            "incidentes": []
        },
        "tipoIncidente": "BACHES",
        "ubicacion": "Av. Principal 123",
        "horaIncidente": "2025-03-23T10:00:00",
        "tipoVialidad": "AVENIDA"
    }
]
\end{lstlisting}

\textit{Notas:}
\begin{itemize}
    \item El parámetro \texttt{\{usuarioId\}} debe coincidir con un ID existente
    \item Devuelve lista vacía si no hay incidentes asociados
\end{itemize}

\textbf{6. Obtener información del usuario (verificación)}

\textit{Endpoint:} \texttt{GET /v1/users/me}

\textit{Descripción:} Obtiene detalles del usuario con contador actualizado.

\textit{Solicitud:}
\begin{itemize}
    \item \textbf{URL:} \texttt{http://localhost:8080/v1/users/me}
    \item \textbf{Método:} \texttt{GET}
    \item \textbf{Headers:} \texttt{correo: juan@example.com}
    \item \textbf{Body:} (ninguno)
\end{itemize}

\textit{Respuesta esperada (éxito, HTTP 200):}
\begin{lstlisting}
{
    "id": 1,
    "nombre": "Juan",
    "apellido": "Perez",
    "correo": "juan@example.com",
    "password": "1234",
    "token": "activo",
    "numeroIncidentes": 1,
    "incidentes": []
}
\end{lstlisting}

\textit{Notas:}
\begin{itemize}
    \item Requiere header \texttt{correo} válido
    \item Verifica estado actual de \texttt{numeroIncidentes}
    \item Muestra información sensible como \texttt{password} (solo para fines demostrativos)
\end{itemize}

\textbf{7. Cerrar sesión (Logout)}

\textit{Endpoint:} \texttt{POST /v1/users/logout}

\textit{Descripción:} Cierra sesión cambiando token a \texttt{"inactivo"}.

\textit{Solicitud:}
\begin{itemize}
    \item \textbf{URL:} \texttt{http://localhost:8080/v1/users/logout}
    \item \textbf{Método:} \texttt{POST}
    \item \textbf{Headers:} \texttt{correo: juan@example.com}
    \item \textbf{Body:} (ninguno)
\end{itemize}

\textit{Respuesta esperada (éxito, HTTP 200):}
\begin{lstlisting}
"Sesion cerrada correctamente"
\end{lstlisting}

\textit{Notas:}
\begin{itemize}
    \item Cambia estado del token a inactivo
    \item Verificación posterior con \texttt{GET /v1/users/me} muestra nuevo estado
    \item Requiere mismo header \texttt{correo} usado en login
\end{itemize}