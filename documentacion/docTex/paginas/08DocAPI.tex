\section*{Documentación de la API REST}

% ================ USUARIOS ================

\subsection*{\faUser\ Endpoints de Usuarios}

\subsection*{\faServer\ 1. Registrar Usuario}
\begin{tcolorbox}[endpoint]
    \textbf{POST} \texttt{/v1/users/create}\\
    Registra un nuevo usuario en el sistema.
\end{tcolorbox}

\begin{tcolorbox}[request]
    \textbf{URL:} \textcolor{urlColor}{\texttt{http://localhost:8080/v1/users/create}}\\
    \textbf{Método:} \textcolor{methodColor}{\texttt{POST}}\\
    \textbf{Headers:} \textcolor{headerColor}{\texttt{Content-Type: application/json}}\\
    \textbf{Body:}
    \begin{minted}[frame=single, bgcolor=gray!10, fontsize=\small]{json}
{
    "nombre": "Juan",
    "apellido": "Pérez",
    "correo": "juan@example.com",
    "password": "secreto123"
}
    \end{minted}
\end{tcolorbox}

\begin{tcolorbox}[response]
    \textbf{Respuesta Exitosa} \faCheckCircle\ \textcolor{successColor}{HTTP 201}
    \begin{minted}[frame=single, bgcolor=gray!10, fontsize=\small]{json}
{
    "mensaje": "Usuario registrado exitosamente",
    "usuario": {
        "id": 1,
        "nombre": "Juan",
        "apellido": "Pérez",
        "correo": "juan@example.com",
        "token": "inactivo",
        "numeroIncidentes": 0,
        "fechaRegistro": "2025-04-26T10:30:00"
    }
}
    \end{minted}
    \textbf{Errores}
    \begin{itemize}
        \item \textcolor{errorColor}{HTTP 400} - Correo electrónico inválido
        \item \textcolor{errorColor}{HTTP 500} - Error interno del servidor
    \end{itemize}
\end{tcolorbox}

\newpage

\subsection*{\faServer\ 2. Iniciar Sesión}
\begin{tcolorbox}[endpoint]
    \textbf{POST} \texttt{/v1/users/login}\\
    Inicia sesión de un usuario en el sistema.
\end{tcolorbox}

\begin{tcolorbox}[request]
    \textbf{URL:} \textcolor{urlColor}{\texttt{http://localhost:8080/v1/users/login}}\\
    \textbf{Método:} \textcolor{methodColor}{\texttt{POST}}\\
    \textbf{Headers:} \textcolor{headerColor}{\texttt{Content-Type: application/json}}\\
    \textbf{Body:}
    \begin{minted}[frame=single, bgcolor=gray!10, fontsize=\small]{json}
{
    "correo": "juan@example.com",
    "password": "secreto123"
}
    \end{minted}
\end{tcolorbox}

\begin{tcolorbox}[response]
    \textbf{Respuesta Exitosa} \faCheckCircle\ \textcolor{successColor}{HTTP 200}
    \begin{minted}[frame=single, bgcolor=gray!10, fontsize=\small]{json}
{
    "mensaje": "Login exitoso",
    "usuario": {
        "id": 1,
        "nombre": "Juan",
        "apellido": "Pérez",
        "correo": "juan@example.com",
        "token": "activo",
        "numeroIncidentes": 0,
        "fechaRegistro": "2025-04-26T10:30:00"
    }
}
    \end{minted}
    \textbf{Errores}
    \begin{itemize}
        \item \textcolor{errorColor}{HTTP 401} - Credenciales inválidas
        \item \textcolor{errorColor}{HTTP 500} - Error en el servidor
    \end{itemize}
\end{tcolorbox}

\newpage

\subsection*{\faServer\ 3. Cerrar Sesión}
\begin{tcolorbox}[endpoint]
    \textbf{POST} \texttt{/v1/users/logout}\\
    Cierra la sesión de un usuario.
\end{tcolorbox}

\begin{tcolorbox}[request]
    \textbf{URL:} \textcolor{urlColor}{\texttt{http://localhost:8080/v1/users/logout}}\\
    \textbf{Método:} \textcolor{methodColor}{\texttt{POST}}\\
    \textbf{Headers:} \textcolor{headerColor}{\texttt{Content-Type: application/json}}\\
    \textbf{Body:}
    \begin{minted}[frame=single, bgcolor=gray!10, fontsize=\small]{json}
{
    "correo": "juan@example.com"
}
    \end{minted}
\end{tcolorbox}

\begin{tcolorbox}[response]
    \textbf{Respuesta Exitosa} \faCheckCircle\ \textcolor{successColor}{HTTP 200}
    \begin{minted}[frame=single, bgcolor=gray!10, fontsize=\small]{json}
{
    "mensaje": "Sesión cerrada correctamente"
}
    \end{minted}
    \textbf{Errores}
    \begin{itemize}
        \item \textcolor{errorColor}{HTTP 404} - Usuario no encontrado
        \item \textcolor{errorColor}{HTTP 500} - Error al cerrar sesión
    \end{itemize}
\end{tcolorbox}

\newpage

\subsection*{\faServer\ 4. Obtener Usuario por Correo}
\begin{tcolorbox}[endpoint]
    \textbf{GET} \texttt{/v1/users/correo/\{correo\}}\\
    Obtiene los datos de un usuario por su correo electrónico.
\end{tcolorbox}

\begin{tcolorbox}[request]
    \textbf{URL:} \textcolor{urlColor}{\texttt{http://localhost:8080/v1/users/correo/juan@example.com}}\\
    \textbf{Método:} \textcolor{methodColor}{\texttt{GET}}
\end{tcolorbox}

\begin{tcolorbox}[response]
    \textbf{Respuesta Exitosa} \faCheckCircle\ \textcolor{successColor}{HTTP 200}
    \begin{minted}[frame=single, bgcolor=gray!10, fontsize=\small]{json}
{
    "mensaje": "Usuario encontrado",
    "usuario": {
        "id": 1,
        "nombre": "Juan",
        "apellido": "Pérez",
        "correo": "juan@example.com",
        "token": "activo",
        "numeroIncidentes": 0,
        "fechaRegistro": "2025-04-26T10:30:00"
    }
}
    \end{minted}
    \textbf{Errores}
    \begin{itemize}
        \item \textcolor{errorColor}{HTTP 404} - Usuario no encontrado
        \item \textcolor{errorColor}{HTTP 500} - Error al buscar usuario
    \end{itemize}
\end{tcolorbox}

\newpage

% ================ INCIDENTES ================

\subsection*{\faExclamationTriangle\ Endpoints de Incidentes}

\subsection*{\faServer\ 5. Crear Incidente}
\begin{tcolorbox}[endpoint]
    \textbf{POST} \texttt{/v1/incidentes/create}\\
    Registra un nuevo incidente en el sistema.
\end{tcolorbox}

\begin{tcolorbox}[request]
    \textbf{URL:} \textcolor{urlColor}{\texttt{http://localhost:8080/v1/incidentes/create}}\\
    \textbf{Método:} \textcolor{methodColor}{\texttt{POST}}\\
    \textbf{Headers:} \textcolor{headerColor}{\texttt{Content-Type: application/json}}\\
    \textbf{Body:}
    \begin{minted}[frame=single, bgcolor=gray!10, fontsize=\small]{json}
{
    "usuarioId": 1,
    "tipoIncidente": "BACHES",
    "ubicacion": "Av. Principal 123",
    "latitud": 19.4326,
    "longitud": -99.1332,
    "tipoVialidad": "AVENIDA"
}
    \end{minted}
\end{tcolorbox}

\begin{tcolorbox}[response]
    \textbf{Respuesta Exitosa} \faCheckCircle\ \textcolor{successColor}{HTTP 201}
    \begin{minted}[frame=single, bgcolor=gray!10, fontsize=\small]{json}
{
    "mensaje": "Incidente creado exitosamente",
    "incidente": {
        "id": "20250426103000_BAC_AVE",
        "usuarioId": 1,
        "tipoIncidente": "BACHES",
        "ubicacion": "Av. Principal 123",
        "latitud": 19.4326,
        "longitud": -99.1332,
        "tipoVialidad": "AVENIDA",
        "estado": "PENDIENTE"
    }
}
    \end{minted}
    \textbf{Errores}
    \begin{itemize}
        \item \textcolor{errorColor}{HTTP 404} - Usuario no encontrado
        \item \textcolor{errorColor}{HTTP 500} - Error al registrar el incidente
    \end{itemize}
\end{tcolorbox}

\newpage

\subsection*{\faServer\ 6. Obtener Todos los Incidentes}
\begin{tcolorbox}[endpoint]
    \textbf{GET} \texttt{/v1/incidentes/all}\\
    Obtiene todos los incidentes registrados en el sistema.
\end{tcolorbox}

\begin{tcolorbox}[request]
    \textbf{URL:} \textcolor{urlColor}{\texttt{http://localhost:8080/v1/incidentes/all}}\\
    \textbf{Método:} \textcolor{methodColor}{\texttt{GET}}
\end{tcolorbox}

\begin{tcolorbox}[response]
    \textbf{Respuesta Exitosa} \faCheckCircle\ \textcolor{successColor}{HTTP 200}
    \begin{minted}[frame=single, bgcolor=gray!10, fontsize=\small]{json}
{
    "mensaje": "Incidentes encontrados",
    "total": 1,
    "incidentes": [
        {
            "id": "20250426103000_BAC_AVE",
            "usuarioId": 1,
            "tipoIncidente": "BACHES",
            "ubicacion": "Av. Principal 123",
            "latitud": 19.4326,
            "longitud": -99.1332,
            "tipoVialidad": "AVENIDA",
            "estado": "PENDIENTE"
        }
    ]
}
    \end{minted}
    \textbf{Errores}
    \begin{itemize}
        \item \textcolor{errorColor}{HTTP 500} - Error al obtener los incidentes
    \end{itemize}
\end{tcolorbox}

\newpage

\subsection*{\faServer\ 7. Obtener Incidentes por Usuario}
\begin{tcolorbox}[endpoint]
    \textbf{GET} \texttt{/v1/incidentes/usuario/\{usuarioId\}}\\
    Obtiene todos los incidentes asociados a un usuario específico.
\end{tcolorbox}

\begin{tcolorbox}[request]
    \textbf{URL:} \textcolor{urlColor}{\texttt{http://localhost:8080/v1/incidentes/usuario/1}}\\
    \textbf{Método:} \textcolor{methodColor}{\texttt{GET}}
\end{tcolorbox}

\begin{tcolorbox}[response]
    \textbf{Respuesta Exitosa} \faCheckCircle\ \textcolor{successColor}{HTTP 200}
    \begin{minted}[frame=single, bgcolor=gray!10, fontsize=\small]{json}
{
    "mensaje": "Incidentes encontrados para el usuario",
    "usuario": {
        "id": 1,
        "nombre": "Juan Pérez"
    },
    "total": 1,
    "incidentes": [
        {
            "id": "20250426103000_BAC_AVE",
            "tipoIncidente": "BACHES",
            "ubicacion": "Av. Principal 123",
            "estado": "PENDIENTE"
        }
    ]
}
    \end{minted}
    \textbf{Errores}
    \begin{itemize}
        \item \textcolor{errorColor}{HTTP 404} - Usuario no encontrado
        \item \textcolor{errorColor}{HTTP 500} - Error al obtener los incidentes
    \end{itemize}
\end{tcolorbox}

\newpage

\subsection*{\faServer\ 8. Actualizar Estado de Incidente}
\begin{tcolorbox}[endpoint]
    \textbf{PUT} \texttt{/v1/incidentes/update-status/\{id\}?usuarioId=\{usuarioId\}}\\
    Actualiza el estado de un incidente específico.
\end{tcolorbox}

\begin{tcolorbox}[request]
    \textbf{URL:} \textcolor{urlColor}{\texttt{http://localhost:8080/v1/incidentes/update-status/20250426103000\_BAC\_AVE?usuarioId=1}}\\
    \textbf{Método:} \textcolor{methodColor}{\texttt{PUT}}\\
    \textbf{Headers:} \textcolor{headerColor}{\texttt{Content-Type: application/json}}\\
    \textbf{Body:}
    \begin{minted}[frame=single, bgcolor=gray!10, fontsize=\small]{json}
{
    "estado": "RESUELTO"
}
    \end{minted}
\end{tcolorbox}

\begin{tcolorbox}[response]
    \textbf{Respuesta Exitosa} \faCheckCircle\ \textcolor{successColor}{HTTP 200}
    \begin{minted}[frame=single, bgcolor=gray!10, fontsize=\small]{json}
{
    "mensaje": "Estado actualizado correctamente",
    "incidente": {
        "id": "20250426103000_BAC_AVE",
        "estado": "RESUELTO"
    }
}
    \end{minted}
    \textbf{Errores}
    \begin{itemize}
        \item \textcolor{errorColor}{HTTP 403} - No tiene permiso para modificar este incidente
        \item \textcolor{errorColor}{HTTP 404} - Incidente no encontrado
        \item \textcolor{errorColor}{HTTP 500} - Error al actualizar el estado
    \end{itemize}
\end{tcolorbox}


\newpage

\subsection*{\faServer\ 9. Buscar Incidentes Cercanos}
\begin{tcolorbox}[endpoint]
    \textbf{GET} \texttt{/v1/incidentes/cercanos}\\
    Busca incidentes cercanos a una ubicación geográfica.
\end{tcolorbox}

\begin{tcolorbox}[request]
    \textbf{URL:} \textcolor{urlColor}{\texttt{http://localhost:8080/v1/incidentes/cercanos?latitud=19.4326\&longitud=-99.1332\&radioKm=5.0}}\\
    \textbf{Método:} \textcolor{methodColor}{\texttt{GET}}\\
    \textbf{Parámetros:}
    \begin{itemize}
        \item \texttt{latitud}: Latitud del punto central
        \item \texttt{longitud}: Longitud del punto central
        \item \texttt{radioKm}: Radio de búsqueda en kilómetros (default: 5.0)
    \end{itemize}
\end{tcolorbox}

\begin{tcolorbox}[response]
    \textbf{Respuesta Exitosa} \faCheckCircle\ \textcolor{successColor}{HTTP 200}
    \begin{minted}[frame=single, bgcolor=gray!10, fontsize=\small]{json}
{
    "mensaje": "Búsqueda completada",
    "parametros": {
        "latitud": 19.4326,
        "longitud": -99.1332,
        "radioKm": 5.0
    },
    "total": 1,
    "incidentes": [
        {
            "id": "20250426103000_BAC_AVE",
            "tipoIncidente": "BACHES",
            "ubicacion": "Av. Principal 123",
            "estado": "PENDIENTE",
            "distancia": 0.5
        }
    ]
}
    \end{minted}
    \textbf{Errores}
    \begin{itemize}
        \item \textcolor{errorColor}{HTTP 500} - Error al buscar incidentes cercanos
    \end{itemize}
\end{tcolorbox}