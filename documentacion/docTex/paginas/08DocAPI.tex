\subsection*{\faServer\ 1. Crear un nuevo usuario}

\begin{tcolorbox}[endpoint]
    \textbf{POST} \texttt{/v1/users/create}\\
    Crea un nuevo usuario en la base de datos.
\end{tcolorbox}

\begin{tcolorbox}[request]
    \textbf{URL:} \textcolor{urlColor}{\texttt{http://localhost:8080/v1/users/create}}\\
    \textbf{Método:} \textcolor{methodColor}{\texttt{POST}}\\
    \textbf{Headers:}\\
    \textcolor{headerColor}{\texttt{Content-Type: application/json}}
    
    \textbf{Body:}
    \begin{minted}[frame=single,
                   framesep=2mm,
                   bgcolor=gray!10,
                   fontsize=\small]{json}
{
    "nombre": "Juan",
    "apellido": "Perez",
    "correo": "juan@example.com",
    "password": "1234",
    "token": "inactivo",
    "numeroIncidentes": 0
}
    \end{minted}
\end{tcolorbox}

\begin{tcolorbox}[response]
    \textbf{Respuesta Exitosa} \faCheckCircle\ \textcolor{successColor}{HTTP 201}
    \begin{minted}[frame=single,
                   framesep=2mm,
                   bgcolor=gray!10,
                   fontsize=\small]{json}
{
    "id": 1,
    "nombre": "Juan",
    "apellido": "Perez",
    "correo": "juan@example.com",
    "password": "1234",
    "token": "inactivo",
    "numeroIncidentes": 0,
    "incidentes": []
}
    \end{minted}
    
    \textbf{Posibles Errores} \faExclamationTriangle
    \begin{itemize}
        \item \textcolor{errorColor}{HTTP 400} - Datos inválidos
        \item \textcolor{errorColor}{HTTP 409} - Correo ya existe
    \end{itemize}
\end{tcolorbox}

\begin{tcolorbox}[notes]
    \faInfoCircle\ \textbf{Notas importantes:}
    \begin{itemize}
        \item El \texttt{id} será generado automáticamente
        \item \texttt{token} inicia como "inactivo"
        \item \texttt{numeroIncidentes} inicia en 0
    \end{itemize}
\end{tcolorbox}

\subsection*{\faServer\ 2. Iniciar sesión (Login)}

\begin{tcolorbox}[endpoint]
    \textbf{POST} \texttt{/v1/users/login}\\
    Inicia sesión y cambia el token a estado "activo".
\end{tcolorbox}

\begin{tcolorbox}[request]
    \textbf{URL:} \textcolor{urlColor}{\texttt{http://localhost:8080/v1/users/login}}\\
    \textbf{Método:} \textcolor{methodColor}{\texttt{POST}}\\
    \textbf{Headers:}\\
    \textcolor{headerColor}{\texttt{Content-Type: application/json}}
    
    \textbf{Body:}
    \begin{minted}[frame=single,
                   framesep=2mm,
                   bgcolor=gray!10,
                   fontsize=\small]{json}
{
    "correo": "juan@example.com",
    "password": "1234"
}
    \end{minted}
\end{tcolorbox}

\begin{tcolorbox}[response]
    \textbf{Respuesta Exitosa} \faCheckCircle\ \textcolor{successColor}{HTTP 200}
    \begin{minted}[frame=single,
                   framesep=2mm,
                   bgcolor=gray!10,
                   fontsize=\small]{json}
{
    "id": 1,
    "nombre": "Juan",
    "apellido": "Perez",
    "correo": "juan@example.com",
    "password": "1234",
    "token": "activo",
    "numeroIncidentes": 0,
    "incidentes": []
}
    \end{minted}
    
    \textbf{Posibles Errores} \faExclamationTriangle
    \begin{itemize}
        \item \textcolor{errorColor}{HTTP 401} - Credenciales incorrectas
        \item \textcolor{errorColor}{HTTP 404} - Usuario no encontrado
    \end{itemize}
\end{tcolorbox}

\begin{tcolorbox}[notes]
    \faInfoCircle\ \textbf{Notas importantes:}
    \begin{itemize}
        \item El token cambiará automáticamente a "activo" tras un login exitoso
        \item La sesión permanecerá activa hasta que se realice logout
        \item Se recomienda almacenar el token para futuras peticiones
    \end{itemize}
\end{tcolorbox}

\subsection*{\faServer\ 3. Crear un nuevo incidente}

\begin{tcolorbox}[endpoint]
    \textbf{POST} \texttt{/v1/incidentes}\\
    Crea un incidente asociado a un usuario específico.
\end{tcolorbox}

\begin{tcolorbox}[request]
    \textbf{URL:} \textcolor{urlColor}{\texttt{http://localhost:8080/v1/incidentes}}\\
    \textbf{Método:} \textcolor{methodColor}{\texttt{POST}}\\
    \textbf{Headers:}\\
    \textcolor{headerColor}{\texttt{Content-Type: application/json}}
    
    \textbf{Body:}
    \begin{minted}[frame=single,
                   framesep=2mm,
                   bgcolor=gray!10,
                   fontsize=\small]{json}
{
    "usuarioId": 1,
    "tipoIncidente": "BACHES",
    "ubicacion": "Av. Principal 123",
    "tipoVialidad": "AVENIDA"
}
    \end{minted}
\end{tcolorbox}

\begin{tcolorbox}[response]
    \textbf{Respuesta Exitosa} \faCheckCircle\ \textcolor{successColor}{HTTP 201}
    \begin{minted}[frame=single,
                   framesep=2mm,
                   bgcolor=gray!10,
                   fontsize=\small]{json}
{
    "id": 1,
    "usuario": {
        "id": 1,
        "nombre": "Juan",
        "apellido": "Perez",
        "correo": "juan@example.com",
        "password": "1234",
        "token": "activo",
        "numeroIncidentes": 1,
        "incidentes": []
    },
    "tipoIncidente": "BACHES",
    "ubicacion": "Av. Principal 123",
    "horaIncidente": "2025-03-23T10:00:00",
    "tipoVialidad": "AVENIDA"
}
    \end{minted}
    
    \textbf{Posibles Errores} \faExclamationTriangle
    \begin{itemize}
        \item \textcolor{errorColor}{HTTP 400} - Datos inválidos en el cuerpo de la solicitud
        \item \textcolor{errorColor}{HTTP 404} - Usuario no encontrado
    \end{itemize}
\end{tcolorbox}

\begin{tcolorbox}[notes]
    \faInfoCircle\ \textbf{Notas importantes:}
    \begin{itemize}
        \item El campo \texttt{usuarioId} debe corresponder a un usuario existente en la base de datos.
        \item El campo \texttt{horaIncidente} será generado automáticamente por el sistema.
        \item El tipo de vialidad (\texttt{tipoVialidad}) debe ser una categoría válida como \texttt{AVENIDA}, \texttt{CALLE}, etc.
    \end{itemize}
\end{tcolorbox}

\subsection*{\faServer\ 4. Recuperar todos los incidentes}

\begin{tcolorbox}[endpoint]
    \textbf{GET} \texttt{/v1/incidentes}\\
    Lista todos los incidentes registrados en la base de datos.
\end{tcolorbox}

\begin{tcolorbox}[request]
    \textbf{URL:} \textcolor{urlColor}{\texttt{http://localhost:8080/v1/incidentes}}\\
    \textbf{Método:} \textcolor{methodColor}{\texttt{GET}}\\
    \textbf{Headers:} (ninguno requerido)\\
    \textbf{Body:} (ninguno)
\end{tcolorbox}

\begin{tcolorbox}[response]
    \textbf{Respuesta Exitosa} \faCheckCircle\ \textcolor{successColor}{HTTP 200}
    \begin{minted}[frame=single,
                   framesep=2mm,
                   bgcolor=gray!10,
                   fontsize=\small]{json}
[
    {
        "id": 1,
        "usuario": {
            "id": 1,
            "nombre": "Juan",
            "apellido": "Perez",
            "correo": "juan@example.com",
            "password": "1234",
            "token": "activo",
            "numeroIncidentes": 1,
            "incidentes": []
        },
        "tipoIncidente": "BACHES",
        "ubicacion": "Av. Principal 123",
        "horaIncidente": "2025-03-23T10:00:00",
        "tipoVialidad": "AVENIDA"
    }
]
    \end{minted}
    
    \textbf{Posibles Errores} \faExclamationTriangle
    \begin{itemize}
        \item \textcolor{errorColor}{HTTP 500} - Error interno del servidor
    \end{itemize}
\end{tcolorbox}

\begin{tcolorbox}[notes]
    \faInfoCircle\ \textbf{Notas importantes:}
    \begin{itemize}
        \item Este endpoint devuelve todos los incidentes registrados en el sistema, sin filtros.
        \item Si no hay incidentes registrados, devuelve una lista vacía (\texttt{[]}).
        \item Se recomienda implementar paginación en caso de grandes volúmenes de datos.
    \end{itemize}
\end{tcolorbox}



\subsection*{\faServer\ 5. Recuperar incidentes por usuario}

\begin{tcolorbox}[endpoint]
    \textbf{GET} \texttt{/v1/incidentes/usuario/\{usuarioId\}}\\
    Lista los incidentes asociados a un usuario específico.
\end{tcolorbox}

\begin{tcolorbox}[request]
    \textbf{URL:} \textcolor{urlColor}{\texttt{http://localhost:8080/v1/incidentes/usuario/1}}\\
    \textbf{Método:} \textcolor{methodColor}{\texttt{GET}}\\
    \textbf{Headers:} (ninguno requerido)\\
    \textbf{Body:} (ninguno)
\end{tcolorbox}

\begin{tcolorbox}[response]
    \textbf{Respuesta Exitosa} \faCheckCircle\ \textcolor{successColor}{HTTP 200}
    \begin{minted}[frame=single,
                   framesep=2mm,
                   bgcolor=gray!10,
                   fontsize=\small]{json}
[
    {
        "id": 1,
        "usuario": {
            "id": 1,
            "nombre": "Juan",
            "apellido": "Perez",
            "correo": "juan@example.com",
            "password": "1234",
            "token": "activo",
            "numeroIncidentes": 1,
            "incidentes": []
        },
        "tipoIncidente": "BACHES",
        "ubicacion": "Av. Principal 123",
        "horaIncidente": "2025-03-23T10:00:00",
        "tipoVialidad": "AVENIDA"
    }
]
    \end{minted}
    
    \textbf{Posibles Errores} \faExclamationTriangle
    \begin{itemize}
        \item \textcolor{errorColor}{HTTP 400} - El parámetro \texttt{\{usuarioId\}} no es válido.
        \item \textcolor{errorColor}{HTTP 404} - Usuario no encontrado o sin incidentes asociados.
    \end{itemize}
\end{tcolorbox}

\begin{tcolorbox}[notes]
    \faInfoCircle\ \textbf{Notas importantes:}
    \begin{itemize}
        \item El parámetro \texttt{\{usuarioId\}} debe coincidir con un ID existente en la base de datos.
        \item Si el usuario no tiene incidentes registrados, el sistema devuelve una lista vacía (\texttt{[]}).
        \item Este endpoint es útil para filtrar incidentes por usuario específico.
    \end{itemize}
\end{tcolorbox}

\subsection*{\faServer\ 6. Obtener información del usuario (verificación)}

\begin{tcolorbox}[endpoint]
    \textbf{GET} \texttt{/v1/users/me}\\
    Obtiene los detalles del usuario, incluyendo el contador actualizado de incidentes.
\end{tcolorbox}

\begin{tcolorbox}[request]
    \textbf{URL:} \textcolor{urlColor}{\texttt{http://localhost:8080/v1/users/me}}\\
    \textbf{Método:} \textcolor{methodColor}{\texttt{GET}}\\
    \textbf{Headers:}\\
    \textcolor{headerColor}{\texttt{correo: juan@example.com}}\\
    \textbf{Body:} (ninguno)
\end{tcolorbox}

\begin{tcolorbox}[response]
    \textbf{Respuesta Exitosa} \faCheckCircle\ \textcolor{successColor}{HTTP 200}
    \begin{minted}[frame=single,
                   framesep=2mm,
                   bgcolor=gray!10,
                   fontsize=\small]{json}
{
    "id": 1,
    "nombre": "Juan",
    "apellido": "Perez",
    "correo": "juan@example.com",
    "password": "1234",
    "token": "activo",
    "numeroIncidentes": 1,
    "incidentes": []
}
    \end{minted}
    
    \textbf{Posibles Errores} \faExclamationTriangle
    \begin{itemize}
        \item \textcolor{errorColor}{HTTP 400} - Header \texttt{correo} no proporcionado o inválido.
        \item \textcolor{errorColor}{HTTP 404} - Usuario no encontrado.
    \end{itemize}
\end{tcolorbox}

\begin{tcolorbox}[notes]
    \faInfoCircle\ \textbf{Notas importantes:}
    \begin{itemize}
        \item Es obligatorio incluir un header \texttt{correo} válido para identificar al usuario.
        \item El campo \texttt{numeroIncidentes} refleja el número actual de incidentes asociados al usuario.
        \item Este endpoint muestra información sensible como \texttt{password} solo para fines demostrativos; en un entorno real, se recomienda omitir este campo.
    \end{itemize}
\end{tcolorbox}

\subsection*{\faServer\ 7. Cerrar sesión (Logout)}

\begin{tcolorbox}[endpoint]
    \textbf{POST} \texttt{/v1/users/logout}\\
    Cierra la sesión del usuario cambiando el estado del token a \texttt{"inactivo"}.
\end{tcolorbox}

\begin{tcolorbox}[request]
    \textbf{URL:} \textcolor{urlColor}{\texttt{http://localhost:8080/v1/users/logout}}\\
    \textbf{Método:} \textcolor{methodColor}{\texttt{POST}}\\
    \textbf{Headers:}\\
    \textcolor{headerColor}{\texttt{correo: juan@example.com}}\\
    \textbf{Body:} (ninguno)
\end{tcolorbox}

\begin{tcolorbox}[response]
    \textbf{Respuesta Exitosa} \faCheckCircle\ \textcolor{successColor}{HTTP 200}
    \begin{minted}[frame=single,
                   framesep=2mm,
                   bgcolor=gray!10,
                   fontsize=\small]{text}
"Sesion cerrada correctamente"
    \end{minted}
    
    \textbf{Posibles Errores} \faExclamationTriangle
    \begin{itemize}
        \item \textcolor{errorColor}{HTTP 400} - Header \texttt{correo} no proporcionado o inválido.
        \item \textcolor{errorColor}{HTTP 404} - Usuario no encontrado.
        \item \textcolor{errorColor}{HTTP 500} - Error interno del servidor.
    \end{itemize}
\end{tcolorbox}

\begin{tcolorbox}[notes]
    \faInfoCircle\ \textbf{Notas importantes:}
    \begin{itemize}
        \item El token del usuario se actualiza automáticamente a \texttt{"inactivo"} tras un logout exitoso.
        \item Es obligatorio incluir el mismo header \texttt{correo} utilizado durante el login.
        \item Se recomienda verificar el estado del token con el endpoint \texttt{GET /v1/users/me} después de cerrar sesión.
    \end{itemize}
\end{tcolorbox}