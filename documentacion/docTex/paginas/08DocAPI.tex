\section*{Documentación de la API REST}

% ================ USUARIOS ================

% --- 1. Registrar Usuario ---
\subsection*{\faServer\ 1. Registrar Usuario}
\begin{tcolorbox}[endpoint]
    \textbf{POST} \texttt{/v1/users/create}\\
    Registra un nuevo usuario en el sistema.
\end{tcolorbox}

\begin{tcolorbox}[request]
    \textbf{URL:} \textcolor{urlColor}{\texttt{http://localhost:8080/v1/users/create}}\\
    \textbf{Método:} \textcolor{methodColor}{\texttt{POST}}\\
    \textbf{Headers:} \textcolor{headerColor}{\texttt{Content-Type: application/json}}\\
    \textbf{Body:}
    \begin{minted}[frame=single, bgcolor=gray!10, fontsize=\small]{json}
{
    "nombre": "Juan",
    "apellido": "Pérez",
    "correo": "juan@example.com",
    "password": "secreto123"
}
    \end{minted}
\end{tcolorbox}

\begin{tcolorbox}[response]
    \textbf{Status:} 201 Created\\
    \textbf{Body:}
    \begin{minted}[frame=single, bgcolor=gray!10, fontsize=\small]{json}
{
    "mensaje": "Usuario registrado exitosamente",
    "usuario": {
        "id": 1,
        "nombre": "Juan",
        "apellido": "Pérez",
        "correo": "juan@example.com",
        "token": "inactivo",
        "numeroIncidentes": 0,
        "fechaRegistro": "2025-05-04T12:00:00"
    }
}
    \end{minted}
\end{tcolorbox}

% --- 2. Login Usuario ---
\subsection*{\faServer\ 2. Login Usuario}
\begin{tcolorbox}[endpoint]
    \textbf{POST} \texttt{/v1/users/login}\\
    Inicia sesión de usuario.
\end{tcolorbox}

\begin{tcolorbox}[request]
    \textbf{URL:} \textcolor{urlColor}{\texttt{http://localhost:8080/v1/users/login}}\\
    \textbf{Método:} \textcolor{methodColor}{\texttt{POST}}\\
    \textbf{Headers:} \textcolor{headerColor}{\texttt{Content-Type: application/json}}\\
    \textbf{Body:}
    \begin{minted}[frame=single, bgcolor=gray!10, fontsize=\small]{json}
{
    "correo": "juan@example.com",
    "password": "secreto123"
}
    \end{minted}
\end{tcolorbox}

\begin{tcolorbox}[response]
    \textbf{Status:} 200 OK\\
    \textbf{Body:}
    \begin{minted}[frame=single, bgcolor=gray!10, fontsize=\small]{json}
{
    "mensaje": "Login exitoso",
    "usuario": {
        "id": 1,
        "nombre": "Juan",
        "apellido": "Pérez",
        "correo": "juan@example.com",
        "token": "activo",
        "numeroIncidentes": 0,
        "fechaRegistro": "2025-05-04T12:00:00"
    }
}
    \end{minted}
\end{tcolorbox}

% --- 3. Logout Usuario ---
\subsection*{\faServer\ 3. Logout Usuario}
\begin{tcolorbox}[endpoint]
    \textbf{POST} \texttt{/v1/users/logout}\\
    Cierra la sesión del usuario.
\end{tcolorbox}

\begin{tcolorbox}[request]
    \textbf{URL:} \textcolor{urlColor}{\texttt{http://localhost:8080/v1/users/logout}}\\
    \textbf{Método:} \textcolor{methodColor}{\texttt{POST}}\\
    \textbf{Headers:} \textcolor{headerColor}{\texttt{Content-Type: application/json}}\\
    \textbf{Body:}
    \begin{minted}[frame=single, bgcolor=gray!10, fontsize=\small]{json}
{
    "correo": "juan@example.com"
}
    \end{minted}
\end{tcolorbox}

\begin{tcolorbox}[response]
    \textbf{Status:} 200 OK\\
    \textbf{Body:}
    \begin{minted}[frame=single, bgcolor=gray!10, fontsize=\small]{json}
{
    "mensaje": "Sesión cerrada correctamente"
}
    \end{minted}
\end{tcolorbox}

% --- 4. Obtener Usuario por ID ---
\subsection*{\faServer\ 4. Obtener Usuario por ID}
\begin{tcolorbox}[endpoint]
    \textbf{GET} \texttt{/v1/users/\{id\}}\\
    Obtiene los datos de un usuario por su ID.
\end{tcolorbox}

\begin{tcolorbox}[request]
    \textbf{URL:} \textcolor{urlColor}{\texttt{http://localhost:8080/v1/users/1}}\\
    \textbf{Método:} \textcolor{methodColor}{\texttt{GET}}
\end{tcolorbox}

\begin{tcolorbox}[response]
    \textbf{Status:} 200 OK\\
    \textbf{Body:}
    \begin{minted}[frame=single, bgcolor=gray!10, fontsize=\small]{json}
{
    "mensaje": "Usuario encontrado",
    "usuario": {
        "id": 1,
        "nombre": "Juan",
        "apellido": "Pérez",
        "correo": "juan@example.com",
        "token": "activo",
        "numeroIncidentes": 0,
        "fechaRegistro": "2025-05-04T12:00:00"
    }
}
    \end{minted}
\end{tcolorbox}

% --- 5. Actualizar datos básicos de usuario ---
\subsection*{\faServer\ 5. Actualizar datos básicos de usuario}
\begin{tcolorbox}[endpoint]
    \textbf{PUT} \texttt{/v1/users/update-datos-basicos}\\
    Actualiza nombre, apellido y/o contraseña del usuario.
\end{tcolorbox}

\begin{tcolorbox}[request]
    \textbf{URL:} \textcolor{urlColor}{\texttt{http://localhost:8080/v1/users/update-datos-basicos}}\\
    \textbf{Método:} \textcolor{methodColor}{\texttt{PUT}}\\
    \textbf{Headers:} \textcolor{headerColor}{\texttt{Content-Type: application/json}}\\
    \textbf{Body:}
    \begin{minted}[frame=single, bgcolor=gray!10, fontsize=\small]{json}
{
    "id": 1,
    "nombre": "Juan",
    "apellido": "Pérez",
    "password": "nuevoPassword123"
}
    \end{minted}
\end{tcolorbox}

\begin{tcolorbox}[response]
    \textbf{Status:} 200 OK\\
    \textbf{Body:}
    \begin{minted}[frame=single, bgcolor=gray!10, fontsize=\small]{json}
{
    "mensaje": "Datos actualizados correctamente"
}
    \end{minted}
\end{tcolorbox}

% --- 6. Crear Incidente ---
\subsection*{\faServer\ 6. Crear Incidente}
\begin{tcolorbox}[endpoint]
    \textbf{POST} \texttt{/v1/incidentes/create}\\
    Registra un nuevo incidente.
\end{tcolorbox}

\begin{tcolorbox}[request]
    \textbf{URL:} \textcolor{urlColor}{\texttt{http://localhost:8080/v1/incidentes/create}}\\
    \textbf{Método:} \textcolor{methodColor}{\texttt{POST}}\\
    \textbf{Headers:} \textcolor{headerColor}{\texttt{Content-Type: application/json}}\\
    \textbf{Body:}
    \begin{minted}[frame=single, bgcolor=gray!10, fontsize=\small]{json}
{
    "usuarioId": 1,
    "tipoIncidente": "BACHES",
    "ubicacion": "Calle 123, Colonia Centro",
    "latitud": 19.4326,
    "longitud": -99.1332,
    "tipoVialidad": "CALLE",
    "fotos": [
        "https://ejemplo.com/foto1.jpg",
        "https://ejemplo.com/foto2.jpg"
    ]
}
    \end{minted}
\end{tcolorbox}

\begin{tcolorbox}[response]
    \textbf{Status:} 201 Created\\
    \textbf{Body:}
    \begin{minted}[frame=single, bgcolor=gray!10, fontsize=\small]{json}
{
    "mensaje": "Incidente creado exitosamente",
    "incidente": {
        "id": "20250504120000_BAC_CAL",
        "usuarioId": 1,
        "tipoIncidente": "BACHES",
        "ubicacion": "Calle 123, Colonia Centro",
        "latitud": 19.4326,
        "longitud": -99.1332,
        "horaIncidente": "2025-05-04T12:00:00",
        "tipoVialidad": "CALLE",
        "estado": "PENDIENTE",
        "fotos": [
            "https://ejemplo.com/foto1.jpg",
            "https://ejemplo.com/foto2.jpg"
        ]
    }
}
    \end{minted}
\end{tcolorbox}

% --- 7. Obtener todos los incidentes ---
\subsection*{\faServer\ 7. Obtener todos los incidentes}
\begin{tcolorbox}[endpoint]
    \textbf{GET} \texttt{/v1/incidentes/all}\\
    Obtiene todos los incidentes registrados.
\end{tcolorbox}

\begin{tcolorbox}[request]
    \textbf{URL:} \textcolor{urlColor}{\texttt{http://localhost:8080/v1/incidentes/all}}\\
    \textbf{Método:} \textcolor{methodColor}{\texttt{GET}}
\end{tcolorbox}

\begin{tcolorbox}[response]
    \textbf{Status:} 200 OK\\
    \textbf{Body:}
    \begin{minted}[frame=single, bgcolor=gray!10, fontsize=\small]{json}
{
    "mensaje": "Incidentes encontrados",
    "total": 2,
    "incidentes": [
        {
            "id": "20250504120000_BAC_CAL",
            "usuarioId": 1,
            "tipoIncidente": "BACHES",
            "ubicacion": "Calle 123, Colonia Centro",
            "latitud": 19.4326,
            "longitud": -99.1332,
            "horaIncidente": "2025-05-04T12:00:00",
            "tipoVialidad": "CALLE",
            "estado": "PENDIENTE",
            "fotos": [
                "https://ejemplo.com/foto1.jpg"
            ]
        }
        // ...otros incidentes
    ]
}
    \end{minted}
\end{tcolorbox}

% --- 8. Obtener incidentes por usuario ---
\subsection*{\faServer\ 8. Obtener incidentes por usuario}
\begin{tcolorbox}[endpoint]
    \textbf{GET} \texttt{/v1/incidentes/usuario/\{usuarioId\}}\\
    Obtiene todos los incidentes de un usuario específico.
\end{tcolorbox}

\begin{tcolorbox}[request]
    \textbf{URL:} \textcolor{urlColor}{\texttt{http://localhost:8080/v1/incidentes/usuario/1}}\\
    \textbf{Método:} \textcolor{methodColor}{\texttt{GET}}
\end{tcolorbox}

\begin{tcolorbox}[response]
    \textbf{Status:} 200 OK\\
    \textbf{Body:}
    \begin{minted}[frame=single, bgcolor=gray!10, fontsize=\small]{json}
{
    "mensaje": "Incidentes encontrados para el usuario",
    "usuario": {
        "id": 1,
        "nombre": "Juan Pérez"
    },
    "total": 1,
    "incidentes": [
        {
            "id": "20250504120000_BAC_CAL",
            "usuarioId": 1,
            "tipoIncidente": "BACHES",
            "ubicacion": "Calle 123, Colonia Centro",
            "latitud": 19.4326,
            "longitud": -99.1332,
            "horaIncidente": "2025-05-04T12:00:00",
            "tipoVialidad": "CALLE",
            "estado": "PENDIENTE",
            "fotos": [
                "https://ejemplo.com/foto1.jpg"
            ]
        }
    ]
}
    \end{minted}
\end{tcolorbox}

% --- 9. Obtener incidente por ID ---
\subsection*{\faServer\ 9. Obtener incidente por ID}
\begin{tcolorbox}[endpoint]
    \textbf{GET} \texttt{/v1/incidentes/\{id\}}\\
    Obtiene un incidente por su ID.
\end{tcolorbox}

\begin{tcolorbox}[request]
    \textbf{URL:} \textcolor{urlColor}{\texttt{http://localhost:8080/v1/incidentes/20250504120000\_BAC\_CAL}}\\
    \textbf{Método:} \textcolor{methodColor}{\texttt{GET}}
\end{tcolorbox}

\begin{tcolorbox}[response]
    \textbf{Status:} 200 OK\\
    \textbf{Body:}
    \begin{minted}[frame=single, bgcolor=gray!10, fontsize=\small]{json}
{
    "mensaje": "Incidente encontrado",
    "incidente": {
        "id": "20250504120000_BAC_CAL",
        "usuarioId": 1,
        "tipoIncidente": "BACHES",
        "ubicacion": "Calle 123, Colonia Centro",
        "latitud": 19.4326,
        "longitud": -99.1332,
        "horaIncidente": "2025-05-04T12:00:00",
        "tipoVialidad": "CALLE",
        "estado": "PENDIENTE",
        "fotos": [
            "https://ejemplo.com/foto1.jpg"
        ]
    }
}
    \end{minted}
\end{tcolorbox}

% --- 10. Actualizar estado de incidente ---
\subsection*{\faServer\ 10. Actualizar estado de incidente}
\begin{tcolorbox}[endpoint]
    \textbf{PUT} \texttt{/v1/incidentes/update-status/\{id\}}\\
    Actualiza el estado de un incidente.
\end{tcolorbox}

\begin{tcolorbox}[request]
    \textbf{URL:} \textcolor{urlColor}{\texttt{http://localhost:8080/v1/incidentes/update-status/20250504120000\_BAC\_CAL?usuarioId=1}}\\
    \textbf{Método:} \textcolor{methodColor}{\texttt{PUT}}\\
    \textbf{Headers:} \textcolor{headerColor}{\texttt{Content-Type: application/json}}\\
    \textbf{Body:}
    \begin{minted}[frame=single, bgcolor=gray!10, fontsize=\small]{json}
{
    "estado": "RESUELTO"
}
    \end{minted}
\end{tcolorbox}

\begin{tcolorbox}[response]
    \textbf{Status:} 200 OK\\
    \textbf{Body:}
    \begin{minted}[frame=single, bgcolor=gray!10, fontsize=\small]{json}
{
    "mensaje": "Estado actualizado correctamente",
    "incidente": {
        "id": "20250504120000_BAC_CAL",
        "usuarioId": 1,
        "tipoIncidente": "BACHES",
        "ubicacion": "Calle 123, Colonia Centro",
        "latitud": 19.4326,
        "longitud": -99.1332,
        "horaIncidente": "2025-05-04T12:00:00",
        "tipoVialidad": "CALLE",
        "estado": "RESUELTO",
        "fotos": [
            "https://ejemplo.com/foto1.jpg"
        ]
    }
}
    \end{minted}
\end{tcolorbox}

% --- 11. Buscar incidentes cercanos ---
\subsection*{\faServer\ 11. Buscar incidentes cercanos}
\begin{tcolorbox}[endpoint]
    \textbf{GET} \texttt{/v1/incidentes/cercanos}\\
    Busca incidentes cercanos a una ubicación geográfica.
\end{tcolorbox}

\begin{tcolorbox}[request]
    \textbf{URL:} \textcolor{urlColor}{\texttt{http://localhost:8080/v1/incidentes/cercanos?latitud=19.4326\&longitud=-99.1332\&radioKm=5.0}}\\
    \textbf{Método:} \textcolor{methodColor}{\texttt{GET}}
\end{tcolorbox}

\begin{tcolorbox}[response]
    \textbf{Status:} 200 OK\\
    \textbf{Body:}
    \begin{minted}[frame=single, bgcolor=gray!10, fontsize=\small]{json}
{
    "mensaje": "Búsqueda completada",
    "parametros": {
        "latitud": 19.4326,
        "longitud": -99.1332,
        "radioKm": 5.0
    },
    "total": 1,
    "incidentes": [
        {
            "id": "20250504120000_BAC_CAL",
            "usuarioId": 1,
            "tipoIncidente": "BACHES",
            "ubicacion": "Calle 123, Colonia Centro",
            "latitud": 19.4326,
            "longitud": -99.1332,
            "horaIncidente": "2025-05-04T12:00:00",
            "tipoVialidad": "CALLE",
            "estado": "RESUELTO",
            "fotos": [
                "https://ejemplo.com/foto1.jpg"
            ]
        }
    ]
}
    \end{minted}
\end{tcolorbox}