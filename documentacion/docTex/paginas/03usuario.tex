\section{Casos de Uso de Bumper}


\subsection{1. Acceso Inicial}
\textbf{Descripción:} El usuario entra a la aplicación y ve una pantalla inicial.\\
\textbf{Flujo:}
\begin{itemize}
    \item Una página principal con un mapa grande en el centro, mostrando puntos de incidentes ya reportados.
    \item Un botón visible que dice ``Reportar Incidente'' para empezar un nuevo reporte.
    \item Un menú en la parte superior con opciones para registrarse, iniciar sesión o seguir como invitado (solo para mirar).
\end{itemize}

\subsection{2. Exploración del Mapa}
\textbf{Descripción:} El usuario navega por el mapa para ver los incidentes.\\
\textbf{Flujo:}
\begin{itemize}
    \item Puede hacer zoom o mover el mapa para explorar diferentes zonas.
    \item Al hacer clic en un punto del mapa, aparece una ventana pequeña con:
    \begin{itemize}
        \item Fotos pequeñas del incidente (hasta 3).
        \item El estatus actual (``No resuelto'', ``En proceso'' o ``Resuelto'').
        \item El tipo de problema (como Bache o Inundación).
    \end{itemize}
    \item A un lado, una barra que se puede abrir o cerrar con:
    \begin{itemize}
        \item Opciones para filtrar por estatus (``Todos'', ``No resuelto'', etc.).
        \item Un menú para elegir el tipo de incidente (como Basura o Alumbrado).
    \end{itemize}
\end{itemize}

\subsection{3. Creación de Reporte}
\textbf{Descripción:} El usuario crea un nuevo reporte paso a paso.\\
\textbf{Flujo:}
\begin{itemize}
    \item Al hacer clic en ``Reportar Incidente'', se abre una barra al lado con un formulario:
    \begin{itemize}
        \item \textbf{Ubicación:} Un mapa pequeño aparece para marcar el lugar exacto; puede escribir una dirección o mover un marcador.
        \item \textbf{Detalles:} 
        \begin{itemize}
            \item Elige el tipo de incidente de una lista (ejemplo: Bache, Basura).
            \item Escribe una breve descripción (hasta 500 caracteres).
            \item Sube de 1 a 3 fotos del problema.
        \end{itemize}
        \item \textbf{Confirmación:} Ve un resumen con el lugar marcado y las fotos, luego elige ``Publicar'' o ``Cancelar''.
    \end{itemize}
    \item Nota: Debe iniciar sesión para poder enviar el reporte.
\end{itemize}

\subsection{4. Post-Publicación}
\textbf{Descripción:} El usuario ve y gestiona su reporte después de enviarlo.\\
\textbf{Flujo:}
\begin{itemize}
    \item El mapa se centra en el nuevo punto reportado.
    \item Al hacer clic en él, aparece una ventana con:
    \begin{itemize}
        \item Un código único para seguir el reporte.
        \item El estatus inicial ``No resuelto''.
        \item Una lista para cambiar el estatus a ``En proceso'' o ``Resuelto'' (solo el creador puede hacerlo).
    \end{itemize}
\end{itemize}

\subsection{5. Interacción de Otros Usuarios}
\textbf{Descripción:} Otros usuarios solo pueden ver información básica de los reportes.\\
\textbf{Flujo:}
\begin{itemize}
    \item Cualquier usuario (registrado o invitado) puede hacer clic en un punto del mapa.
    \item La ventana muestra:
    \begin{itemize}
        \item Las fotos subidas por el creador.
        \item El estatus actual (``No resuelto'', ``En proceso'' o ``Resuelto'').
        \item El tipo de incidente.
    \end{itemize}
    \item No hay opción para añadir fotos ni cambiar el estatus; solo el creador puede modificarlo.
\end{itemize}