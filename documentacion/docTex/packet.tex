%---------------------------- COMANDOS PORTADA ------------------/
%---------------------------------------------------------------/
%\newcommand*{\nombre}{Marco Silva Huerta}
\newcommand*{\fecha}{17 de Febrero de 2025}
\newcommand*{\semestre}{2025-2}
%\newcommand*{\id}{0000000}
\newcommand*{\nombreTarea}{Especificación de Requerimientos}
%\newcommand*{\subtitulo}{}
%---------------------------- COMANDOS PORTADA ----------------\
%---------------------------------------------------------------\


%-----------------------------------INICIO DE PACKETES-------------------/
%-----------------------------------------------------------------------/
\usepackage{amsmath}   % Matemáticas: Comandos extras(cajas ecuaciones) |
\usepackage{amsthm}
\usepackage{amssymb}   % Matemáticas: Símbolos matemáticos              |
\usepackage{datetime}  % Agregar fechas                                 |
\usepackage{graphicx}  % Insertar Imágenes                              |
\usepackage{multicol}  % Creación de tablas                             |
\usepackage{longtable} % Tablas más largas                              |
\usepackage{xcolor}    % Permite cambiar colores del texto              |
\usepackage{tcolorbox} % Cajas de color                                 |
\usepackage{setspace}  % Usar espacios                                  |
\usepackage{fancyhdr}  % Para agregar encabezado y pie de página        |
\usepackage{lastpage}  %                                                |
\usepackage{float}     % Flotantes                                      |
\usepackage{soul}      % "Efectos" en palabras                          |
\usepackage{hyperref}  % Para usar hipervínculos                        |
\usepackage{caption}   % Utilizar las referencias                       |
\usepackage{subcaption} % Poder usar subfiguras                         |
\usepackage{multirow}  % Nos permite modificar tablas                   |
\usepackage{array}     % Permite utilizar los valores para multicolumn  |
\usepackage{booktabs}  % Permite modificar tablas                       |
\usepackage{diagbox}   % Diagonales para las tablas                     |
\usepackage{colortbl}  % Color para tablas                              |
%\usepackage{listings}  % Escribir código                               |
\usepackage{mathtools} % SIMBOLO :=                                     |
\usepackage{enumitem}  % Modificar items de Listas                      |
\usepackage{tikz}      %                                                |
\usepackage{lipsum}    % for auto generating text                       |
\usepackage{afterpage} % for "\afterpage"s                              |
\usepackage{pagecolor} % With option pagecolor={somecolor or none}|     |
\usepackage{xpatch}    % Color de lineas C & F                          |
%\usepackage{glossaries} %                                              |
\usepackage{lastpage}  %                                                |
\usepackage{csquotes}  %                                                | 
\usepackage{array} % Paquete para definir el ancho de las columnas
%-----------------------------------------------------------------------\
%-----------------------------------FIN--- DE PACKETES-------------------\



%----------------------------------------------------------------------/
%-------------------Encabezado y Pie de Pagina-----------------------/ |
%--------------------------------------------------------------------\ |
%\fancyhf{}           %                                                |
%\pagestyle{fancy}

\fancypagestyle{firstpage}{  
    \fancyhead[L]{}
    \fancyhead[R]{}     
    \fancyfoot[L]{Equipo:\textsc{Bumper}}
    \fancyfoot[C]{}
    \fancyfoot[R]{\thepage\ de \pageref*{LastPage}}    
    \renewcommand{\headrulewidth}{0pt} 
    \xpretocmd\headrule{}{}{\PatchFailed}
}

\fancypagestyle{fancy}{  

    \fancyhead[L]{Requerimientos}
    \fancyhead[R]{}     

    \fancyfoot[L]{Equipo:\textsc{Bumper}}
    \fancyfoot[C]{}
    \fancyfoot[R]{\thepage\ de \pageref*{LastPage}}

    \renewcommand{\headrulewidth}{1pt} 
    \xpretocmd\headrule{}{}{\PatchFailed}
    \renewcommand{\footrulewidth}{1.5pt} 
    \xpretocmd\footrule{}{}{\PatchFailed}
}

%--------------------------------------------------------------------\ |
%-------------------Encabezado y Pie de Pagina-----------------------/ |
%------------------------------------------------------------FIN----/


\usepackage{tikz,times}
\usepackage{verbatim}
\usetikzlibrary{mindmap,trees,backgrounds}

%--------------------------------------------------------------------/
%------------------- LISTA DE COLORES ------------------------------/ 
\definecolor{ProcessBlue}{RGB}{0,176,240}
\definecolor{NavyBlue}{RGB}{0,110,184}
\definecolor{Cyan}{RGB}{0,174,239}
\definecolor{MidnightBlue}{RGB}{0,103,49}
\definecolor{ForestGreen}{RGB}{0,155,85}
\definecolor{Goldenrod}{RGB}{255,223,66}
\definecolor{YellowGreen}{RGB}{152,204,112}
\definecolor{Sepia}{RGB}{103,24,0}
\definecolor{Peach}{RGB}{247,150,90}
\definecolor{CarnationPink}{RGB}{242,130,180}
\definecolor{Fuchsia}{RGB}{140,54,140}
\definecolor{WildStrawberry}{RGB}{238,41,103}

\definecolor{Grass}{RGB}{41,238,53}
\definecolor{Meadow}{RGB}{6,243,67}
\definecolor{jellyfish}{RGB}{109,14,130}
\definecolor{rubber}{RGB}{229,27,232}
\definecolor{bullet}{RGB}{225,31,90}
\definecolor{midnight}{RGB}{31,90,225}
\definecolor{sun}{RGB}{241,152,7}
\definecolor{water}{RGB}{16,229,183}

%------------------- COLORES CÓDIGO -------------------- |

%------------------- COLORES JAVA ---------------------- |
\definecolor{backcolour}{RGB}{6,6,6} 
%\definecolor{backcolour}{RGB}{181,181,181} 
\definecolor{codeclassjava}{RGB}{246,113,59}
\definecolor{codegreen}{RGB}{17,225,48}
\definecolor{codenumizq}{RGB}{17,17,17}
\definecolor{codestringjava}{RGB}{51,240,234}
\definecolor{codesymboljava}{RGB}{255,5,0} 
\definecolor{yellowpoint}{RGB}{244,235,100} 
%------------------- COLORES JAVA ---------------------- |


%------------------- COLORES PYTHON -------------------- |
\definecolor{backcolourPY}{RGB}{6,6,6} 
%\definecolor{backcolour}{RGB}{181,181,181} 
\definecolor{codegreenPY}{RGB}{17,225,48}
\definecolor{codeclassPY}{RGB}{246,113,59}
\definecolor{codenumizq}{RGB}{17,17,17}
\definecolor{codestringPY}{RGB}{90,128,220}
\definecolor{codesymboljava}{RGB}{255,5,0} 
%------------------- COLORES PYTHON -------------------- |


%------------------- COLORES CÓDIGO -------------------- |

%------------------- LISTA DE COLORES -------------------------------\
%---------------------------------------------------------------------\